\section{Discussion}
\label{sec:discussion}

In this section, we discuss some shortcomings of \sysname{} and future works.

\para{Expressiveness}: Our \maybe{} APIs allow an easy integration with existing
stream processing systems. While it follows the operator model, combined with
other operators, this is expressive enough. We've presented three applications
in this paper; and we are implement more application using this framework to
understand the expressiveness better.

\para{Fault-tolerance and stragglers:} Currently our system focuses on
adaptation. We have not explored fault-tolerance. While the servers could locate
within a cluster and uses existing stream processing framework for
fault-tolerance, the clients may fail as well. We do not synchronize all
clients. Therefore, application latency may be introduced by straggller rather
than network delays.

\para{Performance modelling:} We aim to present a generic API to express
adaptation. In this work, we do not make any assumptions about the functions
used in each operator. This forces our profiling to perform an exhaustive
search. If we could have some performance modelling of the
operation~\cite{venkataraman2016ernest}, we may only profile a subset and
interpolate or extrapolate the rest of the configurations. We leave such a
performance modelling as future optimization.

\para{Context detection:} Currently we perform online profiling and update the
profile entirely. Real-world data potentially follows a multi-modal
distribution. One optimization is to detect such context changes and use the
profile that best predicts in the current context (such as indoor video vs
outdoor video).

\para{Predicting bandwidth changes:} Model predict control
(MPC)~\cite{yin2015control} has been explored in video streaming to predict the
bandwidth chagnes. Our system could potentially also make predictions and adapt
according to the prediction results; this would further reduces latency;
although predictions need to be cautious for false positives.

%%% Local Variables:
%%% mode: latex
%%% TeX-master: "sosp17"
%%% End:
