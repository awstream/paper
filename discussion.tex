\section{Discussion and Future work}
\label{sec:discussion}

We have presented \sysname{}, a stream processing system achieving low latency
and high accuracy for the wide area. This section discusses some limitations and
our future work.

\para{Fault-tolerance and failure recovery:} \sysname{} tolerates bandwidth
variation but not network partition or host failure. While the servers within
the DCs can handle faults as existing systems---such as Spark
Streaming~\cite{zaharia2013discretized}---do, edge clients should not be
failure-oblivious. We leave the failure detection and recovery of clients as a
future work.

\para{Profile modelling:} \sysname{} currently do not attempt to model $B(c)$
and $A(c)$. Instead it performs an exhaustive search during the profiling. While
parallelism and sampling techniques offer speed up, \sysname{} can benefit from
using statistical techniques during the profiling. For example, Bayesian
Optimization, as demonstrated by CherryPick~\cite{alipourfard2017cherrypick},
models black-box functions and reduces the search time. We plan to explore this
direction to improve our profiling.

% \para{Expressiveness}: Our \maybe{} APIs allow an easy integration with
% existing stream processing systems. While it follows the operator model,
% combined with other operators, this is expressive enough. We've presented
% three applications in this paper; and we are implement more application using
% this framework to understand the expressiveness better.

\para{Context detection:} Currently the offline profiling uses one
representative training dataset and the online profiling updates the profile
continuously. Real world applications could produce data with a multi-modal
distribution, where the mode changes upon context changes, e.g. indoor video vs
outdoor video. One optimization to \sysname{} is allow multiple profiles for one
application, detect context changes at runtime, and use the profile that best
matches the current data distribution.

\para{Predicting bandwidth changes:} Our runtime adaptation currently does not
predict future bandwidth. While reacting to bandwidth changes is enough to
achieve sub-second latency, if \sysname{} can accurately predict future
bandwidth, we expect further improvements such as no latency spikes or a
simplified probing state. Techniques such as Model predictive control (MPC) have
improved QoE in video streaming~\cite{yin2015control} with throughput
prediction.

%%% Local Variables:
%%% mode: latex
%%% TeX-master: "sosp17"
%%% End:
