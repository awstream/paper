\section{Related Work}
\label{sec:related-work}

The work most closely related to \sysname{} is
JetStream~\cite{rabkin2014aggregation}. Both systems target at wide-area
streaming analytics. JetStream proposes to use aggregation and \textit{manual}
degradation policies to achieve responsiveness in the presence of bandwidth
fluctuation. We've discussed its limitations in \autoref{sec:motivation}.

We divide other related works into stream processing systems, approximate
analytics, WAN-aware systems and adaptive video streaming.

\paraf{Stream processing systems:} Streaming databases, such as
Borealis~\cite{abadi2005design},
TelegraphCQ~\cite{chandrasekaran2003telegraphcq}, are the early academic
explorations. They pioneered the usage of dataflow models with specialized
operators for stream processing. Recent research projects and open-source
systems, such as MillWheel~\cite{akidau2013millwheel},
Storm~\cite{toshniwal2014storm}, Heron~\cite{sanjeev2015twitter}, Spark
Streaming~\cite{zaharia2013discretized}, Apache Flink~\cite{carbone2015apache},
primarily focus on fault-tolerant streaming in the context of a single
cluster. While our work has a large debt to the prior streaming works,
\sysname{} is designed for the wide area and explicitly explore the trade-off
between data fidelity and data freshness. In contrast, these stream processing
systems often use TCP and choose to throttle the source when back pressure
happens.

\para{Approximate analytics:} The idea of degrading computation fidelity for
responsiveness has also been explored in other contexts. For SQL queries, online
aggregation~\cite{hellerstein1997online}, BlinkDB~\cite{agarwal2013blinkdb} and
GRASS~\cite{ananthanarayanan2014grass} speed up queries with partial data based
on a statistical model of SQL operators. For real-time processing,
MEANTIME~\cite{farrell2016meantime} uses approximation to guarantee satisfying
real-time deadlines. For video processing, VideoStorm~\cite{zhang2017live}
optimizes video stream \textit{processing} within lager clusters with
approximation and delay-tolerance for resource management and allocation. The
trade-off between available resource and application accuracy is a common theme
among all these works. \sysname{} targets at wide-area streaming analytics, an
emerging application domain especially with the advent of IoT.

\para{WAN-aware systems:} The main challenge in designing geo-distributed
systems is to cope with high latency and limited bandwidth. While some research
projects, such as Vivace~\cite{cho2012surviving}, assume the network can
prioritize a small amount of critical data to avoid delays under congestion,
most systems reduce data transfer in the first place to avoid congestion. For
example, LBFS~\cite{muthitacharoen2001low} exploits similarities between files
or versions of the same file. Over the past few years, we have seen a plethora
of geo-distributed analytical frameworks:
WANalytics~\cite{vulimiri2015wananlytics}, Geode~\cite{vulimiri2015global},
Iridium~\cite{pu2015low}, Pixida~\cite{kloudas2015pixida},
Clarinet~\cite{viswanathan2016clarinet}, etc. They minimize data movement by
incorporating heterogeneous wide-area bandwidth information into query
optimization and data/task placement. While effective in improving analytics
efficiency, most of these works focus on batch tasks such as MapReduce jobs or
SQL query. Such tasks are often executed once, with little real time
constrain. In contrast, \sysname{} focuses on streaming analytics that process
streams of data continuously in real time.

%% - Pixida~\cite{kloudas2015pixida} minimizes data movement across inter-DC
%% links by solving a traffic minimization problem in data analytics.

%% - Iridium~\cite{pu2015low} optimizes data and task placement jointly to
%% achieve an overall low latency goal.

%% - Clarinet~\cite{viswanathan2016clarinet} incorporate bandwidth information
%% into query optimizer to reduce data transfer.

%% WheelFS~\cite{stribling2009flexible}

%% Geode~\cite{vulimiri2015global} develops input data movement and join
%% algorithm selection strategies to minimize WAN bandwidth usage.

%% WANalytics~\cite{vulimiri2015wananlytics} focuses on batch analytics.

%% SWAG~\cite{hung2015scheduling} coordinates compute task scheduling across
%% DCs.

%% \cite{heintz2015towards} discusses the complex tradeoffs involved in
%% wide-area analytics.

\para{Adaptive video streaming:} Designing a good adaptation strategy to improve
QoE for video streaming has been a hot topic, including research
projects~\cite{yin2015control, sun2016cs2p} and industrial efforts
(HLS~\cite{pantos2016http}, DASH~\cite{sodagar2011mpeg,
  michalos2012dynamic}). Because they optimize QoE for humans, their results are
not directly transferable to wide-area streaming analytics. In comparison,
\sysname{} generalizes adaptation by providing \maybe{} APIs to allows more
custom control over what parameters can be tuned. In this way, \sysname{} can
effectively integrate existing and future techniques in video streaming, such as
video encoding (H.264~\cite{richardson2011h} or VP9~\cite{grange2016vp9}). The
main contribution of \sysname{} here is to provide a system for a wide range of
streaming analytics to benefit from adaptation.

%% TCP, for example, adapts to available bandwidth by controlling the congestion
%% window with AIMD~\cite{jacobson1988congestion}.  Previous work has proposed
%% modifications to TCP for specific contexts. For streaming over TCP, Goel et
%% al.~\cite{goel2008low} proposes adaptive buffer-size tuning. For the cloud,
%% Adaptive TCP~\cite{wu2013adaptive} proposes to modify the congestion control
%% behavior based on flow size for the cloud.

% \para{Scheduling and allocation:} Resource allocations in the cloud is how to
% efficiently allocate tasks.  In the wide area, we face fundamental limit that
% therefore degradation is more effective. For multiple tasks, Existing
% allocations are for resources without considering application trade-offs.
% MediaNet~\cite{hicks2003user} uses both local and online global resource
% scheduling to improve user performance and network utilization, and adapts
% without requiring underlying support for resource
% reservations. VideoStorm~\cite{zhang2017live} primarily focuses on cluster
% resource allocation among video queries. For wide-area, the resource
% allocation is limited. We do not control the capacity; but still we can
% allocate the available bandwidth.

% Performance modeling: CherryPick~\cite{alipourfard2017cherrypick},
% Ernest~\cite{venkataraman2016ernest}.
%% Dolly~\cite{ananthanarayanan2013effective}.
%% \para{Program Synthesis:}

%%% Local Variables:
%%% mode: latex
%%% TeX-master: "sosp17"
%%% End:
