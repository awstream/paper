\section{Related Work}
\label{sec:related-work}

\fixme{Most related works have been added; need to refine the wording.}

The works most closely related to \sysname{}'s problem
is JetStream~\cite{rabkin2014aggregation}. Both works focus on wide-area
streaming analytics. JetStream proposes to use aggregation and \textit{explicit}
degradation to achieve responsiveness in the presence of bandwidth fluctuation.
We've discussed the drawbacks with the manual degradation policies.

We divide other related works into stream processing systems, approximate
analytics, WAN-aware systems, adaptive streaming protocols and resource
allocation.

\paraf{Stream processing systems:} Streaming databases, such as
Borealis~\cite{abadi2005design},
TelegraphCQ~\cite{chandrasekaran2003telegraphcq}, are the early academic
explorations. They pioneered the usage of dataflow models with specialized
operators for stream processing. Recent research projects and open-source
systems, such as MillWheel~\cite{akidau2013millwheel},
Storm~\cite{toshniwal2014storm}], Heron~\cite{sanjeev2015twitter}, Spark
Streaming~\cite{zaharia2013discretized}, Apache Flink~\cite{carbone2015apache},
primary focus on fault-tolerant streaming in the context of a single
cluster. While our work has a large debt to the prior streaming work, \sysname{}
is designed for the wide area and explicitly trades data fidelity for data
freshness. In contrast, these stream processing systems often choose to throttle
the source when backpressure happens.

\para{Approximate analytics:} The idea of degrading computation fidelity for
responsiveness has also been explored in other contexts. For SQL queries, online
aggregation~\cite{hellerstein1997online}, BlinkDB~\cite{agarwal2013blinkdb} and
GRASS~\cite{ananthanarayanan2014grass} speed up queries with partial data based
on a statistical model of SQL operators. For real-time processing,
MEANTIME~\cite{farrell2016meantime} ??. Also for video processing,
VideoStorm~\cite{zhang2017live}.

\para{WAN-aware systems:} There has been recent interest in designing
geo-distributed across multiple data centers. Much of the challenge is around
the high latency and limited bandwidth. Some of them assume certain capabilities
of the wide network, for example, Vivace~\cite{cho2012surviving} is a key-value
storage system but assumes the ability to prioritize messages. Others mainly
optimize systems with the awareness of the wide
area. LBFS~\cite{muthitacharoen2001low} is a network file system that exploits
similarities between files or versions of the same file to save bandwidth across
the wide-area. WheelFS~\cite{stribling2009flexible}
Pixida~\cite{kloudas2015pixida} a scheduler that aims to minimize data movement
across resource constrained links.  Iridium~\cite{pu2015low} reduces the latency
for MapReduce jobs by optimizing data and task
placement. Clarinet~\cite{viswanathan2016clarinet} incorporate wide-area
bandwidth information into the query
optimization. Geode~\cite{vulimiri2015global} develops input data movement and
join algorithm selection strategies to minimize WAN bandwidth
usage. WANalytics~\cite{vulimiri2015wananlytics} focuses on batch
analytics. SWAG~\cite{hung2015scheduling} coordinates compute task scheduling
across DCs. We focus on streaming analytics.

%% \cite{heintz2015towards} discusses the complex tradeoffs involved in
%% wide-area analytics.

\para{Adaptive streaming protocols:} Adaptive video streaming has been a
well-studied topic; especially in video-on-demand, we have
HLS~\cite{pantos2016http}, DASH~\cite{michalos2012dynamic}. For live video, RTMP
is typically used; but suffer the accuracy drop. Recent work also improves
quality of user experience~\cite{yin2015control}, but these adaptive techniques
primarily focus on human consumption. Adaptive TCP~\cite{wu2013adaptive}
modifies the congestion control behavior in TCP based on the flow
size. BBR~\cite{cardwell2017bbr} adapts the transmission rate to the bottleneck
bandwidth. Goel et al.~\cite{goel2008low} proposes an adaptive buffer-size
tuning that reduces the latency.

\para{Scheduling and allocation:} MediaNet~\cite{hicks2003user} uses both local
and online global resource scheduling to improve user performance and network
utilization, and adapts without requiring underlying support for resource
reservations. VideoStorm~\cite{zhang2017live} primarily focuses on cluster
resource allocation among video queries. For wide-area, the resource allocation
is limited. We do not control the capacity; but still we can allocate the
available bandwidth.

%% Dolly~\cite{ananthanarayanan2013effective}.

%% \para{Program Synthesis:}

%%% Local Variables:
%%% mode: latex
%%% TeX-master: "sosp17"
%%% End:
