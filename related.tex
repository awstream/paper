\section{Related Work}
\label{sec:related-work}

We divide prior related work into stream processing systems, approximate
analytics, WAN-aware system, adaptive streaming and scheduling.

\paraf{Stream processing systems:} Streaming databases, such as
Borealis~\cite{abadi2005design},
TelegraphCQ~\cite{chandrasekaran2003telegraphcq}, are the early academic
explorations. They pioneered the usage of dataflow models with specialized
operators for stream processing. Recent research projects and open-source
systems, such as MillWheel~\cite{akidau2013millwheel},
Storm~\cite{toshniwal2014storm}], Heron~\cite{sanjeev2015twitter}, Spark
Streaming~\cite{zaharia2013discretized}, Apache Flink~\cite{carbone2015apache},
primary focus on fault-tolerant streaming in the context of a single
cluster. While our work has a large debt to the prior streaming work, \sysname{}
is designed for the wide area and explicitly trades data fidelity for data
freshness. In contrast, these stream processing systems often choose to throttle
the source when backpressure happens.

\para{Approximate analytics:} The idea of degrading computation fidelity for
responsiveness has also been explored in other contexts. For SQL queries, online
aggregation~\cite{hellerstein1997online}, BlinkDB~\cite{agarwal2013blinkdb} and
GRASS~\cite{ananthanarayanan2014grass} speed up queries with partial data based
on a statistical model of SQL operators. For real-time processing,
MEANTIME~\cite{farrell2016meantime} ??. Also for video processing,
VideoStorm~\cite{zhang2017live}. JetStream~\cite{rabkin2014aggregation} is the
closest to our work that studies streaming analytics in wide area and uses
approximation to address the bandwidth scarcity.

\para{WAN-aware:} Historically, researches extend systems from LAN to WAN, and
started to address low bandwidth, such as file system over
WAN~\cite{muthitacharoen2001low}. Then system researches focuses on a single
cluster.  Recently we've seen a return of WAN-aware
research. Vivace~\cite{cho2012surviving} is a key-value storage system for web
applications that span many geographically-distributed sites; it assumes the
ability to prioritize messages in the network. We do not assume. They are
targeted at strong consistency and replicates, we target at streaming analytics
that can tolerate accuracy drop.


 Geode~\cite{vulimiri2015global} and
WANalytics~\cite{wanalytics2015vulimiri} focuses on batch analytics.
\cite{heintz2015towards} discusses the complex tradeoffs involved in wide-area
analytics. JetStream~\cite{rabkin2014aggregation} is perhaps the first work
focuses on wide-area streaming analytics. It proposes to use aggregation and
\textit{explicit} degradation to achieve responsiveness in the presence of
bandwidth fluctuation.

Pixida~\cite{kloudas2015pixida} a scheduler that aims to minimize data movement
across resource constrained links.  GDA~\cite{pu2015low} minimizes latency.
Clarinet~\cite{viswanathan2016clarinet} proposes to execute queries at the data
sites and incorporate bandwidth information into the query optimization.

Other recent work have explored low-layer optimizations to improve GDA query
performance. Iridium [35] develops WAN-aware input data and task placement for
two-stage MapReduce jobs. Geode [43] develops input data movement and join
algorithm selection strategies to minimize WAN bandwidth usage.  Finally,
Jetstream [36] proposes using adaptive filtering and local aggregation of data
to improve latency. SWAG~\cite{hung2015scheduling} coordinates compute task
scheduling across DCs.

\para{Adaptive streaming protocols:} Adaptive video streaming has been a
well-studied topic; especially in video-on-demand, we have
HLS~\cite{pantos2016http}, DASH~\cite{michalos2012dynamic}. For live video, RTMP
is typically used; but suffer the accuracy drop. Recent work also improves
quality of user experience~\cite{yin2015control}, but these adaptive techniques
primarily focus on human consumption. Adaptive TCP~\cite{wu2013adaptive}
modifies the congestion control behavior in TCP based on the flow
size. BBR~\cite{cardwell2017bbr} adapts the transmission rate to the bottleneck
bandwidth. Goel et al.~\cite{goel2008low} proposes an adaptive buffer-size
tuning that reduces the latency.

\para{Scheduling and allocation:} MediaNet~\cite{hicks2003user} uses both local
and online global resource scheduling to improve user performance and network
utilization, and adapts without requiring underlying support for resource
reservations. VideoStorm~\cite{zhang2017live} primarily focuses on cluster
resource allocation among video queries. For wide-area, the resource allocation
is limited. We do not control the capacity; but still we can allocate the
available bandwidth.

%% Dolly~\cite{ananthanarayanan2013effective}.


%% \para{Program Synthesis:}

%%% Local Variables:
%%% mode: latex
%%% TeX-master: "sosp17"
%%% End:
