\section{Related Work}
\label{sec:related-work}

We divide prior related work into stream processing systems, approximate
analytics, WAN-aware system, adaptive streaming and scheduling.

\paraf{Stream processing systems:}
Spark-Streaming~\cite{zaharia2013discretized},
MillWheel~\cite{akidau2013millwheel}, Storm~\cite{toshniwal2014storm}, Apache
Flink~\cite{carbone2015apache} are systems for large-scale stream processing
within datacenters. While they differ from each in the model of computation
(batch vs. streaming), they primarily focus on fault-tolerance; and assume
sufficient bandwidth among worker nodes. Our work aims at low latency across the
wide area; once data arrive at a central location, we could use these systems
for processing.

\para{Approximate analytics:} The idea of trading application accuracy with
approximate has been explored elsewhere, such as database
(BlinkDB~\cite{agarwal2013blinkdb}), real-time processing
(MEANTIME~\cite{farrell2016meantime}), video processing
(VideoStorm~\cite{zhang2017live}). Fully embracing the idea of resource-accuracy
trade-off, this paper explores how to use approximate analytics in the face of
bandwidth scarcity.

\para{WAN-aware:} Historically, researches extend systems from LAN to WAN, and
started to address low bandwidth, such as file system over
WAN~\cite{muthitacharoen2001low}. Then system researches focuses on a single
cluster.  Recently we've seen a return of WAN-aware
research. Vivace~\cite{cho2012surviving} is a key-value storage system for web
applications that span many geographically-distributed sites; it assumes the
ability to prioritize messages in the network. We do not assume. They are
targeted at strong consistency and replicates, we target at streaming analytics
that can tolerate accuracy drop. GDA~\cite{pu2015low},
Clarinet~\cite{viswanathan2016clarinet} proposes to execute queries at the data
sites and incorporate bandwidth information into the query
optimization. Geode~\cite{vulimiri2015global} and
WANalytics~\cite{wanalytics2015vulimiri} focuses on batch analytics.
\cite{heintz2015towards} discusses the complex tradeoffs involved in wide-area
analytics. JetStream~\cite{rabkin2014aggregation} is perhaps the first work
focuses on wide-area streaming analytics. It proposes to use aggregation and
\textit{explicit} degradation to achieve responsiveness in the presence of
bandwidth fluctuation.

\para{Adaptive streaming protocols:} Adaptive video streaming has been a
well-studied topic; especially in video-on-demand, we have
HLS~\cite{pantos2016http}, DASH~\cite{michalos2012dynamic}. For live video, RTMP
is typically used; but suffer the accuracy drop. Recent work also improves
quality of user experience~\cite{yin2015control}, but these adaptive techniques
primarily focus on human consumption. Adaptive TCP~\cite{wu2013adaptive}
modifies the congestion control behavior in TCP based on the flow
size. BBR~\cite{cardwell2017bbr} adapts the transmission rate to the bottleneck
bandwidth. Goel et al.~\cite{goel2008low} proposes an adaptive buffer-size
tuning that reduces the latency.

\para{Scheduling and allocation:} MediaNet~\cite{hicks2003user} uses both local
and online global resource scheduling to improve user performance and network
utilization, and adapts without requiring underlying support for resource
reservations. VideoStorm~\cite{zhang2017live} primarily focuses on cluster
resource allocation among video queries. For wide-area, the resource allocation
is limited. We do not control the capacity; but still we can allocate the
available bandwidth.

%% Dolly~\cite{ananthanarayanan2013effective}.


%% \para{Program Synthesis:}

%%% Local Variables:
%%% mode: latex
%%% TeX-master: "sosp17"
%%% End:
