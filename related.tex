\section{Related Work}
\label{sec:related-work}

We divide prior related work into stream processing systems, approximate
analytics, WAN-aware system, adaptive streaming and scheduling.

\paraf{Stream processing systems:}
Spark-Streaming~\cite{zaharia2013discretized},
MillWheel~\cite{akidau2013millwheel}, Storm~\cite{toshniwal2014storm} are
systems for large-scale stream processing within datacenters.  All these systems
rely on an underlying fault-tolerant storage system, respectively HDFS,
BigTable, or a reli- able message queue. but are orthogonal to our concerns of
efficiency and low latency across the wide area.

Borealis~\cite{abadi2005design},
Storm~\cite{toshniwal2014storm}, Streaming~\cite{zaharia2013discretized}.
MacroBase


\para{Approximate analytics:} BlinkDB~\cite{agarwal2013blinkdb} for database
operations BlinkDB [4] deploys sampling-based approximations on top of MapRe-
duce and Hive to reduce latency. In BlinkDB, the data is carefully pre-sampled
probing jobs are used to estimate query run-time. In con- trast, streaming
wide-area analytics systems such as ours have to measure and adapt to available
bandwidth, with- out the benefit of a prior data-import step. We also support a
range of degradation techniques, not just sampling. with specific statistical
goals; small

and adaptive video streaming~\cite{yin2015control}.

energy and timeliness with approximate computing~\cite{farrell2016meantime}

\para{WAN-aware:} Clarinet~\cite{viswanathan2016clarinet} a novel WAN-aware
query optimizer. Deriving a WAN-aware QEP requires working jointly with the
execution layer of analytics frameworks that places tasks to sites and performs
scheduling. GDA~\cite{pu2015low} on geo-distributed data analytics, but not on
streaming. JetStream~\cite{rabkin2014aggregation} studies streaming analytics in
wide area network. Using structured storage for data aggregation and explicit
degradation, it demonstrates how to achieve responsiveness in the presence of
bandwidth fluctuation. However, JetStream didn't explore how to automatically
synthesize the degradation strategy for each application.

WAN-aware file system~\cite{muthitacharoen2001low}.

Vivace~\cite{cho2012surviving} is a key-value storage system for web
applications that span many geographically-distributed sites; it assumes the
ability to prioritize messages in the network. We do not assume. They are
targeted at strong consistency and replicates, we target at streaming analytics
that can tolerate accuracy drop.

\para{Adaptive Streaming:} Argue that these are only for video viewing (not even
video processing) and the difference. Also HLS~\cite{pantos2016http},
DASH~\cite{michalos2012dynamic} works best with video-on-demand. For live video,
RTMP is typically used; but suffer the accuracy drop.

Adaptive TCP~\cite{wu2013adaptive}. BBR~\cite{cardwell2017bbr}, low latency
adaptive streaming ~\cite{goel2008low}.

\para{Scheduling:} VideoStorm~\cite{zhang2017live},
Dolly~\cite{ananthanarayanan2013effective}.
MediaNet~\cite{hicks2003user}. MediaNet uses both local and online global
resource scheduling to improve user performance and network utilization, and
adapts without requiring underlying support for resource reservations.

%% \para{Program Synthesis:}

%%% Local Variables:
%%% mode: latex
%%% TeX-master: "sosp17"
%%% End:
