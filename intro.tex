\section{Introduction}

%% Background
Wide-area streaming analytics are becoming pervasive, especially with the
emerging class of Internet of Things (IoT) applications.  Large cities such as
London and Beijing have deployed millions of cameras for surveillance and
traffic control~\cite{london.surveillance, skynet}. Buildings are increasingly
equipped with a wide variety of sensors to improve energy efficiency, occupant
comfort, reliability and
maintenance~\cite{krioukov2012building}. Geo-distributed infrastructures, such as
content delivery networks (CDNs), analyze user requests from machine logs over the
globe~\cite{mukerjee2015practical}. These applications need to transport,
distill and analyze streams of data across the wide area in real time.

Most existing stream processing systems, such as
Storm~\cite{toshniwal2014storm}, Spark Streaming~\cite{zaharia2013discretized},
and VideoStorm~\cite{zhang2017live}, targets to work within a single datacenter.
While they are capable of handling large streams of data, they assume the
network is not the bottleneck, which is usually the case for high-bandwidth
datacenter networks. However, the bandwidth in the wide-area network (WAN) is
significantly limited~\cite{hsieh17gaia, vulimiri2015global}. Recent studies
show that the growth of WAN bandwidth has been decelerating for many
years~\cite{global2016telegeography} while traffic demands grows at a staggering
rate~\cite{index2013zettabyte}.

Back-hauling all the data is impractical and inefficient. Recent
works on WAN-aware systems promote pushing computations towards the
edge~\cite{satyanarayanan2009case, rabkin2014aggregation, pu2015low}. However,
communications are not entirely avoidable for the following reasons: $(i)$ some
analytics jobs require joining or aggregating data from multiple geo-distributed
sites~\cite{pu2015low, viswanathan2016clarinet}; $(ii)$ the edge benefits
significantly from central computing resources such as GPUs and
TPUs~\cite{abadi2016tensorflow} in the cloud; $(iii)$ end-devices such as cameras
and mobile devices suffer from the bandwidth in last-hop wireless
links~\cite{zhang2015design, abari2017enabling}.

% could improve efficiency (such as GDA pushes queries out; cloudlet
% etc). However, we still need data transmission for cloud off-loading or
% aggregation purpose.

When facing insufficient bandwidth, application developers need to make a
decision within the design space of data freshness and data fidelity
(\autoref{fig:intro}).

\begin{figure}
  \centering
  \includegraphics[width=0.9\columnwidth]{figures/figure1a.pdf}
  \caption{The trade-off space between data freshness and data fidelity when
    facing insufficient bandwidth. Data collected for our augmented reality
    application. \question{How detailed should we explain the experiment setup?}
    Manual policies are derived from~\cite{rabkin2014aggregation};
    application-specific optimizations is PD's profile for AR. We defer the
    details on how data is collected to \autoref{sec:evaluation}.}
  \label{fig:intro}
  \vspace{-1em}
\end{figure}

First, applications over existing protocols without adaptation often result in one
extreme design point. Streaming over TCP ensures a reliable
delivery but the backlogged data increases application latency. Streaming over
UDP minimizes the latency by sending packets as fast as possible, but
uncontrolled loss devastates application accuracy.

Second, manual policies, such as sampling, allow developers to trade data
fidelity for freshness~\cite{rabkin2014aggregation}. But it's non-trivial to
write accurate policies without expertise or developer effort. In practice,
these policies can result in heuristics without measurements backing up, leading
to sub-optimal performance for both freshness and fidelity.

Third, application-specific optimizations do not generalize. A fine-tuned
algorithm for one application will work poorly for another application, if
performance metrics or data distributions change. Video streaming, for
example, has primarily focused on improving quality of
experience~\cite{yin2015control}. Because humans favor smoothness over image
quality, video streaming maintains a high frame rate, e.g., 25 FPS. This
restriction is unnecessary for other applications, such as machine-based video
analytics that maximize accuracy.

We present \sysname{}, a stream processing system that
simultaneously achieves low latency and high accuracy with minimal developer
effort. The key idea is to build an accurate and precise performance model,
instead of relying on manual or application-specific policies. \sysname{}'s
solution is three-fold: easy-to-use APIs, an automatic profiler and a
low-latency runtime.

\sysname{} augments existing stream processing operators with a new \maybe{}
operator. The basic form of \maybe{} takes a list of values as a knob and a
function that degrades the input stream. The knob specifies different levels of
data degradation that \textit{may} be applied in the degradation function.  We
extend the basic form with a library of specialized operators for common data
types, such as \texttt{maybe\_downsample} for images. Our APIs are simple,
modular and extensible. Developers do not need to be an expert in the
application domain as the knobs tolerate approximate specifications. Multiple
operators can be chained to form a configuration that affects the adaptation
jointly. Arbitrary functions including external libraries can be embedded with
our operators.

\sysname{} then uses a data-driven approach to automatically build application
performance profiles with minimal developer effort. The profiles accurately
capture the relationship between application accuracy and bandwidth consumption
under different combinations of data degradation operations. We use an offline
process to bootstrap our system with developer-supplied training data, and
continuously refine the profiles online to handle potential model drifts. We
exploit parallelism and sampling-based profiling to efficiently explore the
configuration space and learn a Pareto-optimal adaptation strategy.

At runtime, \sysname{} achieves low latency by matching data rate to available
bandwidth, and high accuracy by using Pareto-optimal configurations from the
profile. Upon network congestion, our rate adaptation algorithm increases the
degradation level to reduce bandwidth demand, such that no persistent queue
builds up. To recover, it progressively decreases the degradation level after
probing for more available bandwidth. The runtime also provides additional
options for developers to control application behaviors, e.g., limiting the
maximum allowed bandwidth to reduce WAN cost. For multiple applications, the
profiles allows bandwidth allocation among competing tasks for utility-fairness,
i.e. maximizing the minimal accuracy.

To evaluate \sysname{}, we have built three streaming applications: pedestrian
detection (PD), augmented reality (AR), and monitoring log analysis (LA). We use
real-world data to profile these applications and evaluate their runtime
performance on a geo-distributed public cloud. Our contributions and
evaluation results are summarized as follows.

\begin{itemize}[leftmargin=16pt]

\item We propose a set of \maybe{} operators to incorporate adaptation with
  existing stream processing model. Our programming abstraction is simple,
  modular and extensible.

\item We show that \sysname{}'s data-driven approach generates an accurate and
  precise profile for each application. Parallelism and partial profiling can
  speed up the profiling substantially, up to 29X and 8X respectively.

\item Using runtime experiments on geo-distributed EC2 nodes, \sysname{}
  achieves low latency and high accuracy simultaneously for all applications:
  sub-second latency and 5\% accuracy drop for video analytics, 4-second latency
  and 6\% accuracy drop for LA.

\end{itemize}

%%% Local Variables:
%%% mode: latex
%%% TeX-master: "sosp17"
%%% End:

%% LocalWords: VideoStorm, analytics, CDN, IoT
