\section{Introduction}

In this paper, we study streaming analytics in the wide area. Although recent
stream processing systems (such as Storm~\cite{toshniwal2014storm}, Spark
Streaming~\cite{zaharia2013discretized}) can handle large streams of data in a
single cluster, they are not designed to work in the wide area where the
bandwidth is not sufficient to backhaul all the data to a central location.

With the emerging class of Internet of Things (IoT) applications, such as video
surveillance and industrial monitoring, huge data volumes of data are generated
at the edge. In contrast, the Internet's available bandwidth is scarce and
varying, making it impossible to back-haul all the data. And the demand is
growing faster than the network capacity.

When the network resources become insufficient, applications have to choose
between data freshness and data fidelity. Applications based on TCP ensures a
reliable delivery of the data but the backlogged data will increase application
latency. Applications based on UDP could minimize latency by sending packets as
fast as possible, the uncontrolled packet loss along the network may devastate
the application. Either option is not ideal for many streaming anlaytics.

Previous work (JetStream~\cite{rabkin2014aggregation}) has explored the
direction of reducing application's demand with degradation, but it relies on
developers' manual policy, which lacks precision and faces scalability issue.
There are also application-specific optimizations; but they do not
generalize. For example, adaptive video streaming~\cite{yin2015control} is a
well-studied topic but many adaptations aim at human consumption, focusing on
Quality of Experience (QoE). This limits the adaptation space e.g. maintain
25FPS.

Making degradation practical and general across applications is challenging. The
goal is to maximizing application accuracy under the constrain of available
bandwidth with minimal developers' effort. It calls for the right level of
abstraction with a system-level approach.

In this work, we present \sysname{}, which aims to empower developers with an
easy-to-use framework for wide-area streaming analytics. We focus on the scarce
and variable bandwidth and solve it with three ideas: (1) a novel set of APIs
for structured adaptation; (2) a combination of offline and online profiling to
construct a Pareto-optimal strategy; (3) a rate-based congestion control for
runtime adaptation.

Our proposed API imposes a structure on the adaptation that each application can
perform; although the API is narrow, when combined with other operators, we find
the framework is expressive enough for many streaming analytics.

To liberate developers from specifying rules manually, our system employs a
data-driven empirical-analysis approach. The profiling is done both offline and
online. Offline profiling offers bootstrap information that makes online
profiling more efficient. Online profiling alleviates the problem of
\textit{model drift}.

The runtime adaptation automatically adjusts the applications' execution such
that the streaming demand matches the available bandwidth. We employs a
congestion-based congestion control scheme by measuring the bottleneck
bandwidth; in steady state, the system probes for more available bandwidth.

Using \sysname{}, we've built three applications: pedestrian detection,
augmented reality and distributed top-k. We use real-world data for these
applications to evaluate our systems.

The evaluation shows the generated profile for the three applications. Then for
each application, We show how they adapt the behavior at runtime. Under a
controlled experiment, even with only transient network capacity drop, our
system is able to maintain an end-to-end delay for 1 seconds in the wide-area
and accuracy level above 80\%. Application-agnostic protocols creates
significant backlogged data (TCP for about 100 seconds) or unusable accuracy
(UDP).

In summary, this paper makes the following contributions:

\begin{itemize}[leftmargin=16pt]
\item We study in depth of wide-area streaming applications in the case of
  network resource variation.
\item We propose a novel set of APIs to allow for structured adaptation: they
  require minimal developer efforts while being precise with automatic
  profiling.
\item We propose a new congestion control scheme that adapts the application
  data to available bandwidth, with a goal of minimizing the latency.
\item We implement a prototype system.
\item We build three real-world applications and evaluate their behavior
  under different scenarios.
\end{itemize}

%%% Local Variables:
%%% mode: latex
%%% TeX-master: "sosp17"
%%% End:
