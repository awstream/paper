\section{Introduction}

In this paper, we study streaming analytics in the wide area. Although recent
stream processing systems (such as Storm~\cite{toshniwal2014storm}, Spark
Streaming~\cite{zaharia2013discretized}) can handle large streams of data in a
single cluster, they are not designed to work in the wide area where the
bandwidth is not sufficient to backhaul all the data to a central location.

% Video Surveillance, Electric Grid Monitoring and Distributed Log Analysis,
% etc. Huge data volumes at the edge. video surveillance, 3 mbps per camera [H264
% Primer]. eletrical grid monitoring, 1.4 million data points per second. machine
% logs, 300 million records per day. Scarce and varying bandwidth, impossible to
% backhaul all the data. And the demand is growing faster than the network
% capacity.

When the network resources become insufficient, applications have to choose
between data freshness and data fidelity. Applications based on TCP ensures a
reliable delivery of the data but the backlogged data will increase application
latency. Applications based on UDP could minimize latency by sending packets as
fast as possible, the uncontrolled packet loss along the network may devastate
the application. JetStream~\cite{rabkin2014aggregation}.

Application-specific optimizations do not generalize. For example, adaptive
video streaming~\cite{yin2015control} is a well-studied topic but many
adaptations aim at human consumption, focusing on Quality of Experience
(QoE). This limits the adaptation space e.g. maintain 25FPS.

Making degradation practical is challenging: Goal: Minimize bandwidth while
maximizing application accuracy. Application-specific optimizations don't
generalize. It requires expertise and manual work to explore multidimensional
degradations. The adaptation needs to happen at the runtime: no viable system
yet.

In this work, we present \sysname{}. We tackle this problem with (1) maybe
operators to express degradation (2) automatically learn Pareto-optimal strategy
with multi-dimensional exploration and (3) runtime adaptation balances the
latency with data accuracy.

We've built three applications: pedestrian detection, augmented reality and
distributed top-k.

Evaluation shows that...

In summary, this paper makes the following contributions:

\begin{itemize}[leftmargin=16pt]
\item We study in depth of wide-area streaming applications in the case of
  network resource variation.
\item We propose a novel set of APIs to allow for structured adaptation: they
  require minimal developer efforts while being precise with automatic
  profiling.
\item We propose a new congestion control scheme that adapts the application
  data to available bandwidth, with a goal of minimizing the latency.
\item We implement a prototype system.
\item We build three real-world applications and evaluate their behavior
  under different scenarios.
\end{itemize}

%%% Local Variables:
%%% mode: latex
%%% TeX-master: "sosp17"
%%% End:
