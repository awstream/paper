\subsection{Runtime Adaptation}
\label{sec:runtime}

\begin{figure}
  \centering
  \resizebox{\columnwidth}{!}{
    \begin{tikzpicture}
  %
  % Define basic styles
  %
  % Node
  \tikzstyle{module} = [draw, very thick, rounded corners,
  fill=white, minimum height=2.5em, inner sep=0.5em,
  rectangle, font={\bfseries}, align=center]
  \tikzstyle{cmodule} = [module, fill=black!20]

  % Edge
  \tikzstyle{stateEdgePortion} = [black, thick];
  \tikzstyle{dataEdge} = [stateEdgePortion, ->];
  \tikzstyle{controlEdgePartial} = [stateEdgePortion, dashed];
  \tikzstyle{controlEdge} = [controlEdgePartial, ->];
  \tikzstyle{controlDoubleEdge} = [controlEdgePartial, <->];
  \tikzstyle{edgeLabel} = [pos=0.5, text centered, font={\itshape}];

  \node[name=client, draw, very thick, fill=white,
  double copy shadow={shadow xshift=2pt, shadow yshift=-2pt, fill=white, draw},
  text height=13em, text width=31.5em] {};

  \node[name=server, draw, very thick, fill=white, draw,
  right=of client.north east, anchor=north west, xshift=1em,
  text height=7em, text width=12.5em] {};

  \node[module, name=source, below right=of client.north west,
  xshift=-1.5em, yshift=1.5em] {Source};
  \node[module, name=transform, right of=source, xshift=3.5em] {Transform};
  \node[cmodule, name=degrade, right of=transform, xshift=3.7em] {Degrade};
  \node[cmodule, name=queue, right of=degrade, xshift=3em] {Queue};
  \node[cmodule, name=socket, right of=queue, xshift=3em, text width=3em] {Socket};
  \node[cmodule, name=receiver, right of=socket, xshift=7em] {Receiver};
  \node[module, name=analytics, right of=receiver, xshift=3em] {Analytics};

  \node[cmodule, name=cc, at=($(queue)!0.5!(socket)$), yshift=-6em]
  {Congestion\\Controller};

  \draw ($(source.east)$) edge[dataEdge] ($(transform.west)$);
  \draw ($(transform.east)$) edge[dataEdge] ($(degrade.west)$);
  \draw ($(degrade.east)$) edge[dataEdge] ($(queue.west)$);
  \draw ($(queue.east)$) edge[dataEdge] ($(socket.west)$);
  \draw ($(socket.east)$) edge[dataEdge] node[edgeLabel, yshift=0.6em] {data}
  ($(receiver.west)$);
  \draw ($(receiver.east)$) edge[dataEdge] ($(analytics.west)$);

  %% Control path
  \draw let
  \p1 = ($(cc.center)$), \p2 = ($(degrade.center)$)
  in ($(cc.west)$) edge[controlEdgePartial] (\x2, \y1)
  (\x2, \y1) edge[controlEdge] ($(degrade.south)$);

  \draw let
  \p1 = ($(queue.south)$), \p2 = ($(cc.north)$)
  in ($(\x1, \y1) + (1em,0)$) edge[controlEdge] ($(\x1, \y2) + (1em,0)$);

  \draw let
  \p1 = ($(socket.south)$), \p2 = ($(cc.north)$)
  in ($(\x1, \y1) + (-1em,0)$) edge[controlDoubleEdge] ($(\x1, \y2) + (-1em,0)$);

  \node[name=clientlabel, above right=of client.south west, xshift=-2em, yshift=-2em] {Edge (Client)};
  \node[name=clientlabel, above left=of server.south east, xshift=2em, yshift=-2em] {Server};

  %%
  %% Legend
  %%
  \node[name=datalegend, below=1.5em of server.south west, xshift=2em]
  {\small Data Plane};
  \draw ($(datalegend.west) + (-2em, 0)$) edge[dataEdge]
  ($(datalegend.west) + (-0.5em, 0)$);

  \node[name=controllegend, below=2.5em of datalegend.west, anchor=west]
  {\small Control Plane};
  \draw ($(controllegend.west) + (-2em, 0)$) edge[controlEdge]
  ($(controllegend.west) + (-0.5em, 0)$);

  \node[name=applegend, right=3em of datalegend.east]
  {\small Application Logic};
  \node[module, name=applegendbox, left=0.1em of applegend, text width=0.3em,
  minimum height=0em] {};

  \node[name=syslegend, below=2.5em of applegend.west, anchor=west]
  {\small Runtime};
  \node[cmodule, name=syslegendbox, left=0.1em of syslegend, text width=0.3em,
  minimum size=0em] {};

\end{tikzpicture}

%%% Local Variables:
%%% mode: latex
%%% TeX-master: "sosp17"
%%% End:

  }
  % \includegraphics[width=\linewidth]{figures/runtime-adaptation.pdf}
  \caption{Runtime adaptation system architecture. The grey components are what
    \sysname{} provides.}
  \label{fig:runtime}
\end{figure}

At runtime, the user program is automatically converted to a client half and
server half; and \sysname{} abstracts the communication as well as rate
adaptation. \autoref{fig:runtime} shows our runtime architecture.

The data source with degradation is a module that supports \texttt{update}
function. To handle insufficient bandwidth, an object-level queue bridges data
generation (source) and the network IO. Followed by the queue is a socket module
that abstracts the network communication. It transmits data as fast as possible
and also supports traffic probing. The growth of the queue indicates congestions
and will trigger the congestion controller. The socket module estimates actual
available bandwidth. To avoid spikes in the bandwidth measurement, exponential
smoothing is employed.

The congestion controller (\autoref{fig:cc}) similar to the bottleneck congestion
control~\cite{cardwell2017bbr}. Different from the pacing used in BBR, our
system directly matches the data rate with measured bottleneck bandwidth. In the
probing phase, to achieve a fine-granularity, the congestion controller inserts
additional dummy packets for the purpose of probing additional bandwidth. When
the congestion is resolved.

\begin{figure}
  \centering
  \resizebox{\columnwidth}{!}{
    \begin{tikzpicture}[
  state/.style = { draw, very thick, fill=white, rounded corners=1em,
    minimum height=3em, minimum width=7em, node distance=7em, font={\bfseries},
    align=center },
  edge portion/.style = { black, thick },
  transition/.style = { edge portion, -> },
  algorithm/.style = { draw, thin, fill=white },
  ]

  \node [state] (startup) {
    STARTUP };
  \node [state] (congestion) [right=of startup] {CONGESTION};
  \draw [transition] (startup) -- (congestion)
  node [midway, auto] {Q.Congestion};

  \node [state] (steady) [below=of congestion] {STEADY};

  \draw [transition] ($(congestion.south west)!0.4!(congestion.south east)$)
  to node[midway, sloped, below] {Q.NoQueue}
  ($(steady.north west)!0.4!(steady.north east)$);

  \draw [transition] ($(steady.north west)!0.6!(steady.north east)$)
  to node[midway, sloped, below] {Q.Congestion}
  ($(congestion.south west)!0.6!(congestion.south east)$);

  \node [state] (probe) [left=of steady] {PROBE};

  \draw [transition] ($(steady.south west)!0.6!(steady.north west)$)
  -- ($(probe.south east)!0.6!(probe.north east)$)
  node[midway, auto, swap] {Q.Probe};

  \draw [transition, <-] ($(steady.south west)!0.4!(steady.north west)$)
  -- ($(probe.south east)!0.4!(probe.north east)$)
  node[midway, auto, align=left] {Q.Congestion | \\ IO.ProbeDone};

\end{tikzpicture}


%%% Local Variables:
%%% mode: latex
%%% TeX-master: "sosp17"
%%% End:

  }
  \caption{Congestion Control Algorithm}
  \label{fig:cc}
\end{figure}

%%% Local Variables:
%%% mode: latex
%%% TeX-master: "sosp17"
%%% End:
