\documentclass{sig-alternate-10pt}

\usepackage[hyphens]{url}
\usepackage{bold-extra}
\usepackage{color}
\usepackage{courier}
\usepackage{enumitem}
\usepackage{etoolbox}
\usepackage{microtype}
\usepackage{hyperref}
\usepackage{listings}
\usepackage{subcaption}
\usepackage{times}

\lstset{
  showspaces=false,
  showtabs=false,
  breaklines=true,
  showstringspaces=false,
  breakatwhitespace=true,
  escapeinside={(*@}{@*)},
  basicstyle=\scriptsize\ttfamily,
  columns=fullflexible,
  morekeywords={maybe_downsample, maybe_skip}
}

\setlength{\abovecaptionskip}{4pt}

\def\Snospace~{\S{}}
\renewcommand*\sectionautorefname{\Snospace}
\renewcommand*\subsectionautorefname{\Snospace}
\renewcommand*\subsubsectionautorefname{\Snospace}
\renewcommand*{\figureautorefname}{Fig.}

\begin{document}

\newcommand{\sysname}{AdaptiveStream}
\newcommand{\para}[1]{\smallskip\noindent\textbf{#1}}
\newcommand{\paraf}[1]{\smallskip\noindent\textbf{#1}}
\newcommand{\todo}[1]{{\color{ACMRed}\bf{TODO: #1}\normalfont}}

\def\Snospace~{\S{}}
\renewcommand*\sectionautorefname{\Snospace}
\renewcommand*\subsectionautorefname{\Snospace}
\renewcommand*\subsubsectionautorefname{\Snospace}
\renewcommand*{\equationautorefname}{Eq.}
\renewcommand*{\figureautorefname}{Fig.}

%%% Local Variables:
%%% mode: latex
%%% TeX-master: "sosp17"
%%% End:


% \conferenceinfo{WOODSTOCK}{'97 El Paso, Texas USA}
% \CopyrightYear{2007}
% \crdata{0-12345-67-8/90/01}
% --- End of Author Metadata ---

\title{Adaptive Wide-area Streaming Analytics \\ for the Internet of Things}

\numberofauthors{1}
\author{ \alignauthor Paper \#97, 12 pages }

\maketitle

\begin{abstract}
  In this paper, we present \sysname{}, a stream processing system for real-time
  analytical applications in the wide area. To cope with the scarce and varying
  available bandwidth, \sysname{} allows applications to trade-off accuracy for
  data freshness. To achieve this, \sysname{} first performs an offline
  profiling that explores different degradation strategies and generates an
  accuracy-bandwidth profile. During runtime, \sysname{} adapts the application
  to match the rate of measured bandwidth.

  We evaluate \sysname{} against three real-world applications including
  surveillance applications, augmented reality and distributed log analysis. At
  places where traditional approaches would lead to either significant
  application accuracy drop or long tail latency, our system gracefully
  maintains the balance between application accuracy and system performance.
\end{abstract}

% \category{H.4}{Information Systems Applications}{Miscellaneous}
% \terms{Theory}
% \keywords{ACM proceedings, \LaTeX, text tagging}

\section{Introduction}

%% Background
Wide-area streaming analytics are becoming pervasive, especially with the
emerging class of Internet of Things (IoT) applications.  Large cities such as
London and Beijing have deployed millions of cameras for surveillance and
traffic control~\cite{skynet, london.surveillance}. Buildings are increasingly
equipped with a wide variety of sensors to improve energy efficiency and
occupant comfort~\cite{krioukov2012building}. Geo-distributed infrastructure,
such as content delivery networks (CDNs), analyzes requests from machine logs
over the globe~\cite{mukerjee2015practical}. These applications need to
transport, distill and process streams of data across the wide area in real
time.

Although existing stream processing systems, such as
Storm~\cite{toshniwal2014storm}, Spark Streaming~\cite{zaharia2013discretized},
and VideoStorm~\cite{zhang2017live}, can handle large streams of data, they are
designed to work within a single cluster, where the network is not the
bottleneck.  In contrast, the wide area network (WAN) has limited
bandwidth~\cite{hsieh17gaia, vulimiri2015global}. Moreover, WAN bandwidth growth
has been decelerating for many years~\cite{global2016telegeography} while
traffic demands are growing at a staggering rate~\cite{index2013zettabyte}.

Limited WAN bandwidth makes it neither practical nor efficient to back-haul all
data to a central location.  Recent research on WAN-aware systems promotes
pushing computations towards the edge~\cite{pu2015low, rabkin2014aggregation,
  satyanarayanan2009case}. However, communication is not entirely avoidable:
$(i)$ some analytical jobs require joining or aggregating data from multiple
geo-distributed sites~\cite{pu2015low, viswanathan2016clarinet}; $(ii)$ the edge
benefits substantially from central computing resources such as GPUs and
TPUs~\cite{abadi2016tensorflow} in the cloud; and $(iii)$ end-devices such as
cameras and mobile devices suffer from limited bandwidth in last-hop wireless
links when running processing on nearby edge
infrastructure~\cite{abari2017enabling, zhang2015design}.

% could improve efficiency (such as GDA pushes queries out; cloudlet
% etc). However, we still need data transmission for cloud off-loading or
% aggregation purpose.

When facing insufficient bandwidth, application developers need to make a
decision within the design space of data fidelity versus freshness
(\autoref{fig:intro}).

\begin{figure}
  \centering
  \includegraphics[width=0.8\columnwidth]{figures/figure1.pdf}
  \caption{The trade-off space between data freshness and fidelity when facing
    insufficient bandwidth (details in \autoref{sec:runtime-adaptation}).}
  \label{fig:intro}
  \vspace{-1em}
\end{figure}

Applications using existing protocols without adaptation result in extreme
design points. Streaming over TCP ensures a reliable delivery but backlogged
data increases latency. On the other hand, streaming over UDP minimizes latency
by sending packets as fast as possible, but uncontrolled loss devastates
application accuracy.

Degradation, as demonstrated by JetStream~\cite{rabkin2014aggregation}, allows
developers to trade data fidelity for freshness. While it's easy to write
policies for simple operations, such as sampling, in general, accurate policies
will require extensive expertise and considerable efforts. In practice,
developers write policies based on some set of heuristics rather than
quantitative measurements. These inaccurate manual degradation policies lead to
sub-optimal performance for both freshness and fidelity.

Furthermore, application-specific optimizations do not generalize. A fine-tuned
adaptation algorithm for one application will work poorly for another
application, if performance metrics or data distributions change.  For example,
video streaming focuses on quality of experience
(QoE)~\cite{michalos2012dynamic, pantos2016http, yin2015control}. Because humans
favor smoothness over image quality, video streaming systems maintain a high
frame rate, e.g.\,\(25~\text{FPS}\), and reduce image resolutions under
bandwidth limit.  Adaptation by reducing resolutions is a poor match for
analytic jobs that rely on image details.

To achieve low latency and high accuracy simultaneously with minimal developer
effort, we design and implement \sysname{}, a stream processing system for the
wide area. The key idea is to build an accurate and precise performance model
instead of relying on manual or application-specific policies. \sysname{}'s
solution is three-fold: easy-to-use APIs, automatic profiling, and a low-latency
runtime.

\sysname{} augments existing stream processing operators with a new \maybe{}
operator. Its basic form takes a list of values as a knob and a function that
degrades the input stream. The knob specifies the degradation level that affects
data size and data fidelity. We extend the basic form with a library of
specialized operators for common data types, such as \texttt{maybe\_downsample}
for images. Our APIs are simple, composable and extensible. Developers do not
need to be an expert in the application domain as the knobs tolerate approximate
specifications. Multiple operators form a configuration that affects the
adaptation jointly. Arbitrary functions and external libraries can be embedded
with our operators.

\sysname{} then uses a data-driven approach to automatically build application
performance profiles with minimal developer effort. The profiles accurately
capture the relationship between application accuracy and bandwidth consumption
under different combinations of data degradation operations. We use an offline
process to bootstrap our system with developer-supplied training data, and
continuously refine the profile online to handle model drifts. We exploit
parallelism and sampling-based profiling to efficiently explore the
configuration space and learn a Pareto-optimal adaptation strategy.

At runtime, \sysname{} achieves low latency by matching data rate to available
bandwidth, and high accuracy by using Pareto-optimal configurations from the
profile. Upon network congestion, our rate adaptation algorithm increases the
degradation level to reduce data rate, such that no persistent queue builds
up. To recover, it progressively decreases the degradation level after probing
for more available bandwidth. The runtime also provides additional options for
developers to control application behaviors, e.g., limiting the maximum allowed
WAN bandwidth. For multiple applications, the profiles allow bandwidth
allocation among competing tasks for utility fairness.

To evaluate \sysname{}, we've built three streaming applications: augmented
reality (AR), pedestrian detection (PD), and distributed Top-K analysis (TK). We
use real-world data to profile these applications and evaluate their runtime
performance on a geo-distributed public cloud. Our contributions and evaluation
results are as follows.

\begin{itemize}[leftmargin=*, topsep=2pt, itemsep=2pt]

\item We propose \maybe{} operators to incorporate adaptation with existing
  stream processing models. Our programming abstraction is simple, composable
  and extensible.

\item We show that \sysname{}'s data-driven approach generates an accurate and
  precise profile for each application. Parallelism and sampling techniques
  can speed up the profiling substantially, up to 29$\times$ and 9$\times$\@.

\item Using runtime experiments on geo-distributed EC2 nodes, \sysname{}
  achieves low latency and high accuracy simultaneously for all
  applications---sub-second latency and 2\% accuracy drop for video analytics,
  4-second latency and 1\% accuracy drop for TK\@.

\end{itemize}

%%% Local Variables:
%%% mode: latex
%%% TeX-master: "awstream"
%%% End:

%% LocalWords: VideoStorm, analytics, CDN, IoT

\section{Motivation}
\label{sec:background-motivation}

In this section, we make the case for an adaptive stream processing system in
the wide area by examining the gap between application demands and the existing
infrastructure. We start with a few streaming applications.

\para{Video Surveillance:} We envisage a city-wide monitoring system that
aggregates camera feeds (both stationary ground cameras and moving aerial
vehicles) and analyzes video streams in real-time for surveillance, anomaly
detection or business intelligence~\cite{oh2011large}. While traditionally human
labors are involved in analyzing abnormal activities, recent advances in
computer vision and deep learning has dramatically increased the accuracy for
automatic analysis of visual scenes, such as pedestrian
detection~\cite{dollar2012pedestrian}, vehicle tracking~\cite{coifman1998real},
or facial recognition to locate people of interest~\cite{parkhi2015deep,
  Lu:2015:SHF:2888116.2888245}. \todo{Add concrete numbers to argue for the data
  volume.}

\para{IoT Sensors:} While traditional environmental sensors are
slow~\cite{atzori2010internet}, we are seeing an increasing trend with
high-frequency, high-precision sensors being deployed. For example, uPMU
monitoring system for the electrical grid consists of a network of 1000 devices;
each produces 12 streams of 120 Hz high-precision values accurate to 100
ns. This amounts to 1.4 million points per second that requires specialized
timeseries database~\cite{andersen2016btrdb}.

\para{Log Analysis:} Large organizations today are managing 10--100s of
datacenters (DCs) and edge clusters worldwide~\cite{calder2013mapping}. While
most log analysis today runs in a batch mode and on a daily basis, there is
trend in analyzing logs in real-time for quicker optimization \todo{cite RISE
  reference?}. For example, a content distribution network (CDN) can improve the
overall efficiency by optimizing data placement if the access logs can be
processed in real-time.

\vspace{0.5em}

We consider the practical issues with deploying these applications. While they
challenge the data storage and processing system, the cloud can handle it well.
The real challenge lies in the communication. Data generated from the edge, not
a lot WAN bandwidth; also with cost. And worse, the bandwidth is also not
guaranteed. We will demonstrate with measurement.

\subsection{Wide-Area Bandwidth Characteristics}
\label{sec:making-case-adapt}

\begin{figure}
  \centering
  \includegraphics[width=.95\linewidth]{figures/europe-to-us-west.pdf}
  \caption{Bandwidth measurement between Amazon EC2 sites (from Ireland to
    California).}
  \label{fig:bw}
\end{figure}

To understand the bandwidth characteristics in the wide-area, we conducted a
simple measurement using Amazon EC2. We use iPerf~\cite{iperf} to measure the
pair-wise bandwidth between four geo-distributed sites throughout the day. We
observed large variance in the measured bandwidth and one such pair is shown in
\autoref{fig:bw}. Regardless of the number of flows\footnote{EC2 has a per-flow
  and per-VM rate limiting~\cite{zhang2016guaranteeing}.}, these exist occasions
when the available bandwidth is almost halved. We believe the backhaul links
between EC2 sites are better (if not at least representative) in comparison to
the general wide-area links. The varying nature poses real challegen to the
realization and successful deployment of wide-area streaming applications.

\subsection{Bandwidth-Accuracy Trade-off}
\label{sec:bat}

% Existing stream processing systems in the wide area often directly use TCP as
% their transport. TCP works remarkably well estimating the available bandwidth
% and minizing flow completion time. Although TCP adapts the congestion window in
% dynamically based on the feedback from the
% receiver~\cite{jacobson1988congestion}, being application agnostic, TCP delivers
% whatever the application wants to send; and in the case of limited network
% capacity, TCP creates backlogged data, causing significant delay.

% For applications where TCP's retransmission is not unnessary, UDP is often
% chosen. Many multimedia applications such as Internet
% telephony~\cite{baset2004analysis}. Live IP cameras streaming with RTMP. Or
% sensor data over OSC~\cite{wright1997open}, a protocol based on UDP. UDP's
% packet loss can be detrimental in the deteriorate situations.

% The issue with these protocol is that they are designed to be generally
% applicable to many applications without intervening the application execution.
% Often developers of individual applications need to tune the transport to fit
% their needs~\cite{tierney2001tcp} or deal with the insufficient bandwidth case.

The edge infrastructure is capable of pre-processing the data before the
communication. Data degradation. Such as frame-diff based video encoding
scheme. In the case of the top-K application, we first generate windowed local
counts. In our dataset, we see 100x data size reduction. While effective, these
transformations are often not sufficient. In the case of top-k, there is a long
tail. In the case of images/video, quantizing individual pixels will often give
more space.

They help reduce the resource demand but they typically also lower the output
quality. \autoref{fig:log-trade-off} shows how the image resolution affects
application accuracy.

In some verticle domain, such as video encoding, adaptive scheme
exists. However, there is no general solution and these solutions are not
generally applicable to all applications. Most video encoding techniques will
adjust the quantizer to tune the encoding size and quality; while often
preserving the frame rate as these videos are for human consumption. A smooth
video provides a better quality of experience than higher resolution but
intermitten images.

We presented empirical results with grouped, windowed aggregation on PlanetLab
using Akamai log data, and highlighted the complexity of tradeoffs that we show
are driven by several factors such as query, data, and resource
characteristics. local aggregation and global
aggregation.~\cite{heintz2015towards}

This motivates us to design an application-aware rate-adapting stream processing
framework for the wide area; primarily exploring the design space of
degradation.

For these applications, there is a trade-off between the bandwidth and accuracy.
Exploring the design space that allows explicit trading accuracy for resource
constrained cases is the main goal of this paper.

\begin{figure}
  \centering
  \begin{subfigure}{.48\columnwidth}
    \centering
    \includegraphics[width=.95\linewidth]{figures/motiv-resolution.pdf}
    \label{fig:log-bw}
  \end{subfigure}
  \begin{subfigure}{.48\columnwidth}
    \centering
    \includegraphics[width=.95\linewidth]{figures/motiv-framerate.pdf}
    \label{fig:log-acc}
  \end{subfigure}
  \caption{The trade-off between required bandwidth and the accuracy.}
  \label{fig:log-trade-off}
\end{figure}

%%% Local Variables:
%%% mode: latex
%%% TeX-master: "sigcomm2017"
%%% End:

\section{\sysname{} Overview}
\label{sec:system-overview}

In this section, we present an overview of \sysname{}. The primary goal of
\sysname{} is to enable applications with the ability to adapt its communication
in a guided manner.

\subsection{Challenges}
\label{sec:challenges}

\noindent There are four challenges in realizing \sysname{}.

\para{C1: Diverse application and data:} As discussed in~\autoref{sec:bat}, the
best adaptation scheme is often application- and context-specific
optimizations. It becomes important to separate individual application logic
from specific degradation strategy as well as adaptation mechanism.

\para{C2: No analytical solutions:} Unlike SQL queries whose demand and accuracy
can typically be estimated using analytical models~\cite{cormode2012synopses},
many of our streaming applications are dealing with unstructure data using
either use blackbox operations (such as H.264 encoding) or non-linear operators
(such as thresholding). The impact of these degradations is not immediately
available.

\para{C3: Multi-dimensional exploration:} Real-world applications typically have
more than one tunable parameters; leading to a combinatorial space for
exploration. In addition, these parameters are not necessarily orthognal.  The
optimal degradation strategies may only be achievable when more than one
degradation is in effect.

\para{C4: Runtime adaptation at application layer:} Although recent work on
resource reservation makes it possible to guarantee quality of service with new
IP or MAC layer protocols in LAN (e.g. TSN~\cite{johas2013heterogeneous}), we
target at WAN analytics where most of the infrastructure is owned by others and
shared among many users. An application-layer solution is in favor to those that
require special hardware or software upgrade.

\subsection{System Architecture}
\label{sec:architecture}

\begin{figure*}
  \centering
  \includegraphics[width=\linewidth]{figures/arch.pdf}
  \caption{\sysname{} Overview.}
  \label{fig:overview}
\end{figure*}

To address the aforementioned challenges, \sysname{}'s solution is split into
four parts (\autoref{fig:overview}).

\para{Programming abstraction:} Applications are modelled as a directed acyclic
graph (DAG) of computation and we propose a novel set of \texttt{maybe}
operators to express the specification of degradation. Our propose APIs do not
require developers to be exact on the quantity; integrating this into existing
applications requires minimal effort (\autoref{sec:prog-abs}).

\para{Automatic multi-dimensional profiling:} Our system automatically explores
the parameter space to generate a Pareto-optimal degradation strategy with a
multi-dimensional exploration. We also report our experience and findings about
degradation operations (\autoref{sec:profiling}).

\para{Runtime Adaptation:} Finally the streaming application is deployed with a
wide-area orchestration manager. At runtime, our system generates additional
modules that acts as the control plane to adapt the application execution. The
control plane performs bandwidth estimation, congestion monitoring and
adaptation. It uses the profile generated from the second stage and adjust the
level of degradation (\autoref{sec:adaptation}).

%%% Local Variables:
%%% mode: latex
%%% TeX-master: "sigcomm2017"
%%% End:

\section{Profile}
\label{sec:profile}

Describe how we do profiling in this section.

%%% Local Variables:
%%% mode: latex
%%% TeX-master: "sigcomm2017"
%%% End:

\subsection{Runtime Adaptation}
\label{sec:adaptation}

\begin{figure}
  \centering
  \includegraphics[width=\linewidth]{figures/runtime.pdf}
  \caption{Runtime adaptation system architecture. The grey components are what
    \sysname{} provides.}
  \label{fig:runtime}
\end{figure}

At runtime, applications automatically adapt the degradation level based on the
feedback from receiver (\autoref{fig:runtime}).

\para{Bandwidth Measurement:} The receiver measures application-level througput,
similar to \texttt{iPerf}~\cite{iperf}. To avoid sudden change, we use
exponential smoothing. To avoid unnecessary communication, the client requests
the measurement only when congestion is detected.

\para{Object-level Queue:} A queue where the unit is application object (frame).

\para{Congestion monitor:} Using a queue-based detection with high watermark and
low watermark. This can be fixed queue length, fixed data size, or estimated
queue delay based on the rate. We use the rate based.

\para{Degradation Manager:} It loads the learned profile. When receiving signals
from the congestion monitor, it adjusts the degradation level to match the
measured bandwidth. We leave some headroom here to allow queued objects being
sent. When a congestion cleared message is received, the manager increases the
rate. If some degradation operation is still in effect, after a fixed amount of
time, the manager would try to reduce the level of degradation more; until it
reaches the maximal allowed configuration.

\para{Degrade:} The actual degradation operation is rather simple. Operators
based on the \texttt{maybe} API supports a \texttt{set} function that would
change the internals of the operator. The \texttt{set} function is invoked when
degradation is needed.

%% {Interaction with TCP:} Our runtime system follow the tuning guide
%% \cite{tierney2001tcp} to adjust the send buffer.

%%% Local Variables:
%%% mode: latex
%%% TeX-master: "sigcomm2017"
%%% End:

% \section{System Design}
\label{sec:system-design}

\subsection{Profiling Applications}
\label{sec:prof-appl}

Below is a snippet of applications that uses our API. Each of the `maybe\_X`
function is a design space hinted by the application developer. The `Param` is
another way to get input knowledge from the application developer.

\begin{figure}
  \begin{lstlisting}
source.maybe_skip(Upto(10))
    .maybe_downsample(MaxRate(0.5))
    .maybe_batch(30)
    .maybe_transform(|frame|
        roi(frame, Rect::new(0, 0, 30, 30)))
    .then(|frame|
        hog_detect(frame, Param::range(30, 50)))
    .collect();
  \end{lstlisting}
  \caption{Hello World}
  \label{fig:code}
\end{figure}

We take hints from the application developers who use our API. For each block of
\texttt{MaybeDownsample}, \texttt{MaybeDegrade}, \texttt{MaybeBatch} and
\texttt{Tunable}, we first run an off-line profiling phase that generates the
application profile.

At first sight, this involves a combinatorial search step.

Naively this requires scanning the entire parameter space, which is a
high-dimension space. But we assume that domain experts could embed their
knowledge to assist the search.

\subsection{Per-task Adaptation}
\label{sec:per-task-adaptation}

During the execution of the application, our runtime monitors the available
resource and tune the application according to the offline generated graph.

\subsection{Inter-task Allocation}
\label{sec:inter-task-alloc}

When multiple tasks are available, since their profiles are different, the naive
adaptation (drop half for all) is not idea. We explore how these taks can
cooperatively improve the system.

\subsection{Processing Placement}
\label{sec:processing-placement}

%%% Local Variables:
%%% mode: latex
%%% TeX-master: "sigcomm2017"
%%% End:

\section{Implementation}
\label{sec:implementation}

%%% Local Variables:
%%% mode: latex
%%% TeX-master: "sigcomm2017"
%%% End:

\section{Evaluation}
\label{sec:evaluation}

We evaluate \sysname{} for two parts. First, we demonstrate the effectiveness of
our multi-dimensional profiling. We show the generated profile for the three
applications we've built; the result validates our prior hypothesis on the
degradation impact. Second, we show our runtime adaptation is able to maintain
low latency and high accuracy in the case of severe network degradation. Under a
controlled experiment, even with only transient network capacity drop, our
system is able to maintain an end-to-end delay for ~10 seconds in the wide-area
and accuracy level above 80\%. Application-agnostic protocols creates
significant backlogged data (TCP for about 100 seconds) or unusable accuracy
(UDP).

\subsection{Degradation Performance}
\label{sec:degr-perf}

\begin{figure*}
  \centering
  \begin{subfigure}[t]{0.30\textwidth}
    \centering
    \includegraphics[width=\textwidth]{figures/ped-profile.pdf}
    \caption{Pedestrian Detection}
    \label{fig:pd-profile}
  \end{subfigure}
  ~
  \begin{subfigure}[t]{0.30\textwidth}
    \centering
    \includegraphics[width=\textwidth]{figures/darknet-profile.pdf}
    \caption{Augmented Reality}
    \label{fig:ar-profile}
  \end{subfigure}
  ~
  \begin{subfigure}[t]{0.30\textwidth}
    \centering
    \includegraphics[width=\textwidth]{figures/log-profile.pdf}
    \caption{Top-k}
    \label{fig:tk-profile}
  \end{subfigure}
  \caption{Profile. \todo{add more legend}}
  \label{fig:all-profiles}
\end{figure*}


We describe the dataset we used for offline profiling and interpret the
profiling results (\autoref{fig:all-profiles}) in turn.

\para{Pedestrian Detection:} We use MOT16 dataset~\cite{milan2016mot16} to
evaluate this application. Specifically we used MOT16-04 as the training
dataset. The video feeds capture a busy pedestrian street at night with an
elevated viewpoint. The original resolution is 1920x1080, with frame rate
30. The training data has 1050 frames in total, amounting 35-second monitoring.
On averge there are 45.3 people per frame.

There are three knobs in this application: resolution, frame rate and encoding
quality. To maintain the same 16:9 aspect ratio with the original 1920x1080
resolution, the first degradation only chooses common 16:9 resolutions:
1600x900, 1280x720, 960x540, 640x320. For the framerate, integer values are
chosen in favor of fraction values. The original frame rate is 30, and our
degradation explores 10, 5, 3, 2, 1. H.264 encoding quantizer has a range from 0
(lossless) to 51 (worst possible), and 18 is the visually
lossless~\cite{bellard2012ffmpeg}. In our experiment, we use 10, 20, 30, 40, 50
as degradation parameters.

The generated profile is shown in~\autoref{fig:pd-profile} with x-axis the
required bandwidth and the y-axis the accuracy (F1 score). Note the log scale on
the horizontal axis as raw uncompressed 8-bit RGB video streams are
prohibitively large: $1920 \times 1080 \times 30 \times 3 \times 8 = 1.5 $ Gbps.

Each point in the scatter plot represents one configuration that our offline
profiling has evaluated. Notice the vast spread in bandwidth requirement among
configurations with similar accuracy as well as the wide spread in accuracy
among configurations that consumes similar bandwidth.

We first show three lines in the case of only tuning one knob. Notice the
distinct behavior of the three lines. In this particular application, reducing
the resolution has the most penalty because the HOG detector has a minimal 128
pixels by 64 pixels window. The camera is deployed in a far-field context;
scaling down the image will quickly has an effect on the detection. Tuning frame
rate doesn't affect the accuracy too much. In fact, even with 1 FPS, the
accuracy is still relatively high. However, reducing frame rate doesn't bring
much bandwidth saving (as we have mentioned in \autoref{fig:h264}).  The most
effective way that reduces the bandwidth while preserving the accuracy is to
adjust the quantizer because it affects almost every pixel and creates smaller
P-frames.

The Pareto boundary, or \textit{profile}, is the most important curve. In the
begining it's close to the curve when only quantizer is tuned, quantizer has a
certain limit. A crisp image is prefered as many image processing algorithms are
looking for the edges while for human consumption, a smoother image is fine.

As the uncompressed video is not practical, we imposes a bandwidth cap before
the profile is used in runtime (only use optimal configurations that creates
video with less than 20mbps data rate).

\para{Augmented Reality:} We collected training set for this application
ourselves. It's a 23-second video clip with 1920x1080 resolution and 30 FPS
taken on a mobile phone. During the capture, we change the camera view in a slow
pace to emulate how a real user would look around. Because target objects are
relatively close while the camera is moving, we hypothesis for this training
set, the profile will be different from the previous application that reducing
frame rate will have a detrimental effect.

The generate profile is shown in \autoref{fig:ar-profile}. First, we see our
intuition backed up by measurements. Besides, the Pareto boundary also first
follows the video encoding knob, but optimal settings are achieved only when
multiple degradations are in effect.

\para{Top-K:} To evalute the top-k application, we generate synthetic dataset
based on real-world access logs (EDGAR log file dataset, the access log of
\url{https://sec.gov}). The original log contains CSV-format data extract from
Apache web server that records and stores user access
statistics~\cite{edgarlog}. The original log has only 500k access per hour; it's
rather small in comparison to today's CDN log. We condensed an hour-long data
into one second. After performing the local aggregation, the data size is
reduced from 500k entries per second to 50k key-value pairs (10x reduction).
Next we explore the space of degradation with respect to parameter N and T.  The
parameter N is from 100 to 15000; T from 0 to 500.

\autoref{fig:tk-profile} shows the generated profile. As we can see, most
configurations are very close to the pareto boundary. In the case when data skew
is more severe, we might see that T is more severe. Regardless, with our
automatic profiling tool, developers don't have to thoroughly understand the
complex relationship between bandwidth, accuracy and configuration.

\subsection{Runtime Performance}
\label{sec:runtime-performance}

To evaluate the runtime behavior, we conduct controlled experiments using four
geo-distributed worker nodes from Amazon EC2 (t2.large instances) and an
aggregation server from our institute. For each experiment, worker nodes
transmit test data for about 10 mins. During each session, we use Linux
\texttt{tc} utility to adjust outgoing bandwidth to experiment with network
resource variation.

We compare our system with baseline systems that directly uses TCP and UDP. In
all three applications, the raw data streams are orders of magnitude
larger. While our system can adapt the rate, it could be unfair to baseline
solutions. We adjust the default degradation operation so that TCP and UDP would
work just fine when in normal cases; in this way, we make fair comparison. In
the case of UDP, shaping at the source doesn't emulate the packet loss behavior
with out-of-order delivery. We use \texttt{netem} to control packet loss rate to
match the desired shaping bandwidth.

In all three experiments, we see long delays in TCP. It increases linearly when
the traffic shaping started. When the bandwidth shaping stops, TCP quickly fills
the connection to recover. Depending on the queued size, the recovery could take
a few minutes or tens of seconds.

For UDP, the latency has been consistently small (mostly below 1 second) because
there is no queue building up. But when traffic shaping starts, the accuracy
drop is catastrophic.

Our system is a middle-ground between two baselines.

We notice that our delay is still on the order of ten seconds. The reason for
the slow adaptation is three-folds: (1) our current implementation only requests
for bandwidth information when congestion is detected, the delay of getting
bandwidth estimation can be large in the case of network capacity drop; (2) we
perform the bandwidth in a conservative way with smoothing to avoid sudden
spikes. While more improvements are possible, the current settings are
satisfactory.

\begin{figure*}[t!]
  \centering
  \begin{subfigure}[t]{0.30\textwidth}
    \centering
    \includegraphics[width=0.95\textwidth]{figures/ped-runtime.pdf}
    \caption{Pedestrian Detection}
  \end{subfigure}
  ~
  \begin{subfigure}[t]{0.30\textwidth}
    \centering
    \includegraphics[width=0.95\textwidth]{figures/degrade-placeholder-2.pdf}
    \caption{Object Recognition}
  \end{subfigure}
  ~
  \begin{subfigure}[t]{0.30\textwidth}
    \centering
    \includegraphics[width=0.95\textwidth]{figures/cdn-runtime}
    \caption{Top-k}
  \end{subfigure}
  \caption{Runtime Adapation Behavior of \sysname{}.  \todo{adjust the accuracy
      measure}}
  \label{fig:runtime}
\end{figure*}

%%% Local Variables:
%%% mode: latex
%%% TeX-master: "sigcomm2017"
%%% End:
\section{Related Work}
\label{sec:related-work}

Borealis~\cite{abadi2005design}, Storm~\cite{toshniwal2014storm},
Streaming~\cite{zaharia2012discretized}.

JetStream~\cite{rabkin2014aggregation}.
Clarinet~\cite{viswanathan2016clarinet}, GDA~\cite{pu2015low}

%%% Local Variables:
%%% mode: latex
%%% TeX-master: "sigcomm2017"
%%% End:

\section{Discussion}
\label{sec:discussion}

We have presented \sysname{}, a stream processing system achieving low latency
and high accuracy for the wide area. We then discuss \sysname{}'s limitations
and our future work.

\para{Fault-tolerance and failure recovery:} \sysname{} tolerates bandwidth
variation but not network partition or host failure. While the servers within
the DCs can handle faults as existing systems---such as Spark
Streaming~\cite{zaharia2013discretized}---do, edge clients should not be
failure-oblivious. We leave the failure detection and recovery of clients as a
future work.

\para{Profile modelling:} \sysname{} currently do not attempt to model $B(c)$
and $A(c)$. Instead it performs an exhaustive search during the profiling. While
parallelism and sampling techniques offer speed up, there are other statistical
techniques. For example, Bayesian Optimization, as demonstrated by
CherryPick~\cite{alipourfard2017cherrypick}, models black-box functions and
reduces the search time. We plan to explore this direction to improve our
profiler.

\para{Expressiveness}: Our \maybe{} APIs allow an easy integration with existing
stream processing systems. While it follows the operator model, combined with
other operators, this is expressive enough. We've presented three applications
in this paper; and we are implement more application using this framework to
understand the expressiveness better.

\para{Context detection:} Currently we perform online profiling and update the
profile entirely. Real-world data potentially follows a multi-modal
distribution. One optimization is to detect such context changes and use the
profile that best predicts in the current context (such as indoor video vs
outdoor video).

\para{Predicting bandwidth changes:} Model predict control
(MPC)~\cite{yin2015control} has been explored in video streaming to predict the
bandwidth chagnes. Our system could potentially also make predictions and adapt
according to the prediction results; this would further reduces latency;
although predictions need to be cautious for false positives.

%%% Local Variables:
%%% mode: latex
%%% TeX-master: "sosp17"
%%% End:

\section{Conclusion}
\label{sec:conclusion}

Whew, finally...

%%% Local Variables:
%%% mode: latex
%%% TeX-master: "sigcomm2017"
%%% End:


{\bibliographystyle{acm}\bibliography{sigcomm2017}}

\end{document}

%%% Local Variables:
%%% mode: latex
%%% TeX-master: t
%%% End: