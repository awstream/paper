\subsection{Multitask Adaptation}
\label{sec:multitask-adaptation}

The learned profile is also useful to control network resource allocation; this
are extremely useful in the wide-area where the bandwidth is precious or there
is a budget constrain. For a single stream, developers can set a maximal allowed
data rate. For multiple streams, the profiles allow \textit{utility fairness} in
addition to resource fairness. We elaborate on the multiple streams (multitask)
scenario below.

Traditional resource fairness (such as TCP converges to a equal share of the
bandwidth) among competing flows if they share the bottleneck link. However,
resource fairness is different from utility fairness. Consider two profiles.  An
allocation of X, Y leads to 0.3 and 0.7 accuracy; while another allocation could
achieve 0.6 and 0.6 accuracy.

We formulate the allocation as the following optimization problem: the
congestion controller finds an allocation $c_i^t$ to maximize the
\textit{minimal} application accuracy among all running tasks, subject to that
the total bandwidth demand does not exceed available bandwidth
(\autoref{eq:multitask}).

\begin{equation}
  \label{eq:multitask}
  \begin{aligned}
    & \underset{c_i^t}{\text{maximize}} & & \min({A^t(c_i^t)}) & & \\
    & \text{subject to} & & \sum_t{B^t(c_i^t)} < R & & \\
  \end{aligned}
\end{equation}

Solving this optimization is NP hard. In practice, we use heuristic solutions to
approach the allocation.

\newpage

%%% Local Variables:
%%% mode: latex
%%% TeX-master: "sosp17"
%%% End:
