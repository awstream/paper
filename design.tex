\section{System Design}
\label{sec:system-design}

\subsection{Profiling Applications}
\label{sec:prof-appl}

Below is a snippet of applications that uses our API. Each of the `maybe\_X`
function is a design space hinted by the application developer. The `Param` is
another way to get input knowledge from the application developer.

\begin{figure}
  \begin{lstlisting}
source.maybe_skip(Upto(10))
    .maybe_downsample(MaxRate(0.5))
    .maybe_batch(30)
    .maybe_transform(|frame|
        roi(frame, Rect::new(0, 0, 30, 30)))
    .then(|frame|
        hog_detect(frame, Param::range(30, 50)))
    .collect();
  \end{lstlisting}
  \caption{Hello World}
  \label{fig:code}
\end{figure}

We take hints from the application developers who use our API. For each block of
\texttt{MaybeDownsample}, \texttt{MaybeDegrade}, \texttt{MaybeBatch} and
\texttt{Tunable}, we first run an off-line profiling phase that generates the
application profile.

At first sight, this involves a combinatorial search step.

Naively this requires scanning the entire parameter space, which is a
high-dimension space. But we assume that domain experts could embed their
knowledge to assist the search.

\subsection{Per-task Adaptation}
\label{sec:per-task-adaptation}

During the execution of the application, our runtime monitors the available
resource and tune the application according to the offline generated graph.

\subsection{Inter-task Allocation}
\label{sec:inter-task-alloc}

When multiple tasks are available, since their profiles are different, the naive
adaptation (drop half for all) is not idea. We explore how these taks can
cooperatively improve the system.

\subsection{Processing Placement}
\label{sec:processing-placement}

%%% Local Variables:
%%% mode: latex
%%% TeX-master: "sigcomm2017"
%%% End:
