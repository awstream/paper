\section{Discussion and Future work}
\label{sec:discussion}

We have presented \sysname{}, a stream processing system achieving low latency
and high accuracy for the wide area. This section discusses limitations and
future work.

\para{Reducing developer efforts.} While \sysname{} focuses on reducing
developer efforts from tedious manual policy construction, there are still
application-specific aspects for developers: writing appropriate \maybe{} calls,
providing training data, and specifying the accuracy function. Because
\sysname{}'s API is extensible, we plan to build more libraries for common
degradation operators and accuracy functions, similar to machine learning
libraries.

\para{Fault-tolerance and failure recovery.} \sysname{} tolerates bandwidth
variation but not network partition or host failure. Although the servers within
the DCs can handle faults in existing systems, such as Spark
Streaming~\cite{zaharia2013discretized}, it is difficult to make edge clients
failure-oblivious.  We leave failure detection and recovery as a future work.

\para{Profile modelling.} \sysname{} currently does not model $B(c)$/$A(c)$. It
performs an exhaustive search when profiling. While parallelism and sampling
techniques can speed up, \sysname{} can benefit further from statistical
techniques. Bayesian Optimization~\cite{snoek2012practical}, as demonstrated by
CherryPick~\cite{alipourfard2017cherrypick}, models black-box functions and
speeds up profiling by searching for near-optimal configurations. We plan to
explore this direction to improve our profiling.

% \para{Expressiveness}: Our \maybe{} APIs allow an easy integration with
% existing stream processing systems. While it follows the operator model,
% combined with other operators, this is expressive enough. We've presented
% three applications in this paper; and we are implement more application using
% this framework to understand the expressiveness better.

\para{Context detection.} \sysname{} is currently limited to one profile: the
offline profiling generates the default profile and the online profiling updates
the single profile continuously.  Real-world applications may produce data with
a multi-modal distribution, where the model changes upon context changes, such
as indoor versus outdoor. Therefore, one possible optimization to \sysname{} is
to allow multiple profiles for one application, detect context changes, and use
the profile that best matches the current data distribution.  Switching contexts
could reduce, or even eliminate, the overhead of online profiling.

\para{Predicting bandwidth changes.} Our runtime currently does not predict
future bandwidth. While reacting to bandwidth changes is enough to achieve
sub-second latency, if \sysname{} can accurately predict future bandwidth, we
expect further improvements such as no latency spikes or a simplified runtime
design. Techniques such as model predictive control (MPC) have improved QoE in
video streaming with throughput prediction~\cite{yin2015control}; we plan to
investigate using MPC in \sysname{} as a future work.

%%% Local Variables:
%%% mode: latex
%%% TeX-master: "../awstream"
%%% End:
