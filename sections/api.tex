\subsection{APIs for Structured Adaptation}
\label{sec:struct-adapt}

%% Introduce graphs of operators model
The majority of stream processing applications today are constructed as a
directed graph of operators~\cite{toshniwal2014storm, zaharia2013discretized},
where each operator transforms input streams into new streams. \sysname{}
borrows the same computation model. We list some example operators, such as
\texttt{map} and \texttt{skip}, in \autoref{tab:operators}.

To integrate adaptation as a first-class abstraction, \sysname{} introduces \maybe{} operators
that degrade data quality, yielding potential bandwidth savings.
Our API design has three considerations:
$(i)$~to free developers from specifying exact rules, the API should tolerate
approximate specifications; $(ii)$~to allow combining multiple dimensions, the API
should be modular: each operator is a unit and the developer can chain multiple
operators; $(iii)$~to support flexible integration with arbitrary degradation
functions, the API should take user-defined functions (UDFs). Therefore, our API
is,

\vspace{-2pt}
\begin{lstlisting}
        maybe(knobs: Vec<T>, f: (T, I) => I)
\end{lstlisting}

We illustrate the use of the \texttt{maybe} operator with an example that
quantizes a stream of integers in the Rust programming language:

\begin{table*}
  \footnotesize
  \centering
  \begin{tabular}{ c r l }
    \toprule
    \multirow{5}{*}{Normal Operators}
    & \textit{map} (f: I $\Rightarrow$ O) & Stream<I> $\Rightarrow$ Stream<O> \\
    & \textit{skip} (i: Integer) & Stream<I> $\Rightarrow$
                                   Stream<I> \\
    & \textit{sliding\_window} (count: Integer, f: Vec<I> $\Rightarrow$ O) & Stream<I> $\Rightarrow$
                                                                            Stream<O> \\
    % & \textit{tumbling\_window} (count: Integer, f: Vec<I> $\Rightarrow$ O) & Stream<I> $\Rightarrow$
    %                                                                          Stream<O> \\
    & \textit{timed\_window} (time: Duration, f: Vec<I> $\Rightarrow$ O) & Stream<I> $\Rightarrow$
                                                                          Stream<O> \\
    & ... & ... \\
    \midrule
    \multirow{5}{*}{Degradation Operators}
    & \textit{maybe} (knobs: Vec<T>, f:  (T, I) $\Rightarrow$ I) & Stream<I> $\Rightarrow$
                                                                 Stream<I> \\
    & \textit{maybe\_skip} (knobs: Vec<Integer>) & Stream<I> $\Rightarrow$ Stream<I> \\
    & \textit{maybe\_head} (knobs: Vec<Integer>) & Stream<Vec<I>{}> $\Rightarrow$
                                                   Stream<Vec<I>{}> \\
    & \textit{maybe\_downsample} (knobs: Vec<(Integer, Interger)>) & Stream<Image> $\Rightarrow$ Stream<Image> \\
    & ... & ... \\
    \bottomrule
  \end{tabular}
  \caption{Stream processing operators in \sysname{}. \texttt{Vec<T>} represents
    a list of elements with type \texttt{T}.}
  \label{tab:operators}
\end{table*}

\vspace{-2pt}
\begin{lstlisting}
let quantized_stream = vec![1, 2, 3, 4].into_stream()
  .maybe(vec![2, 4], |k, val| val.wrapping_div(k))
  .collect();
\end{lstlisting}

The snippet creates a stream of integers, chains a degradation operation, and
collects the execution result. In this example, the knob is [2, 4], and the
degradation function performs a wrapping (modular) division where the divisor is
the chosen knob. The knob value modifies the quantization level, affecting the
output: [1, 2, 3, 4] (no degradation), [0, 1, 1, 2] (k=2), or [0, 0, 0, 1]
(k=4). If the stream is then encoded---e.g. run-length encoding as in
JPEG~\cite{wallace1992jpeg}---for transmission, the bandwidth consumption will
change according to the level of degradation.

Based on the \texttt{maybe} primitive, one can implement wrappers of degradation
operations for common data types. For instance, \texttt{maybe\_head} will
optionally takes the top values of a list; and \texttt{maybe\_downsample} can
adjust the image resolution to a configured target. \sysname{} provides a number
of such operations as a library to simplify application development
(\autoref{tab:operators}).

With our APIs, the example mentioned in \autoref{sec:making-case-sys-approach}
can now be implemented as follows:

\vspace{-2pt}
\begin{lstlisting}
  let app = Camera::new((1920, 1080), 30)
    .maybe_downsample(vec![(1600, 900), (1280, 720)])
    .maybe_skip(vec![2, 5])
    .map(|frame| frame.show())
    .compose();
\end{lstlisting}

This snippet first instantiates a \texttt{Camera} source, which produces
\texttt{Stream<Image>} with 1920x1080 resolution and 30 FPS\@. Two degradation
operations follow the source: one that downsamples the image to either 1600x900 or
1280x720 resolution, and one that skips frames with a parameter of 2 or 5, resulting
in 30/(2+1)=10 FPS or 30/(5+1)= 6 FPS\@. This example then shows degraded images
on the display. In practice, operators for further processing, such as encoding and
transmission, can be chained.

%%% Local Variables:
%%% mode: latex
%%% TeX-master: "awstream"
%%% End:
