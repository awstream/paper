\section{Related Work}
\label{sec:related-work}

We first compare \sysname{} with JetStream~\cite{rabkin2014aggregation} and then
divide all other related works into stream processing systems, approximate
analytics, WAN-aware systems, adaptive video streaming, and quality of service
(QoS). \fixme{can remove this paragraph.}

\para{JetStream.} \fixme{JetStream is the closest system to AWStream.} Both \sysname{}
and JetStream target at wide-area streaming
analytics and use degradation to reduce latency. Compared to JetStream,
\sysname{} makes the following major contributions: $(1)$ a novel API design
that simplifies the development of adaptive applications; $(2)$ techniques for
composing degradation across operators in an optimal and automated manner; $(3)$
\fixme{add offline here} online profiling to address model drift; $(4)$ runtime implementation
that
utilizes the learned profile and achieves sub-second latency (comparison made in
\autoref{sec:runtime}). $(5)$ support for different resource allocation policies
for multiple applications.

\para{Stream processing systems.} Streaming databases, such as
Borealis~\cite{abadi2005design},
TelegraphCQ~\cite{chandrasekaran2003telegraphcq}, are the early academic
explorations. They pioneered the use of dataflow models with specialized
operators for stream processing. Recent research projects and open-source
systems, such as MillWheel~\cite{akidau2013millwheel},
Storm~\cite{toshniwal2014storm}, Heron~\cite{sanjeev2015twitter}, Spark
Streaming~\cite{zaharia2013discretized}, Apache Flink~\cite{carbone2015apache},
primarily focus on fault-tolerant streaming in the context of a single
cluster. While our work has a large debt to the prior streaming works,
\sysname{} targets at the wide area and explicitly explores the trade-off
between data fidelity and freshness. In contrast, these stream processing
systems often use TCP and choose to throttle the source when back pressure
happens.

\para{Approximate analytics.} The idea of degrading computation fidelity for
responsiveness is also explored elsewhere. For SQL queries, online
aggregation~\cite{hellerstein1997online}, BlinkDB~\cite{agarwal2013blinkdb} and
GRASS~\cite{ananthanarayanan2014grass} speed up queries with partial data based
on a statistical model of SQL operators. For real-time processing,
MEANTIME~\cite{farrell2016meantime} uses approximation to guarantee satisfying
real-time deadlines. For video processing, VideoStorm~\cite{zhang2017live}
optimizes video stream \textit{processing} within lager clusters by exploiting
approximation and delay-tolerance for resource management and allocation. The
trade-off between available resource and application accuracy is a common theme
among all these works. \sysname{} targets at wide-area streaming analytics, an
emerging application domain especially with the advent of IoT\@.

\para{WAN-aware systems.} The main challenge in designing geo-distributed
systems is to cope with high latency and limited bandwidth. While some research
projects, such as Vivace~\cite{cho2012surviving}, assume the network can
prioritize a small amount of critical data to avoid delays under congestion,
most systems reduce data transfer in the first place to avoid congestion. For
example, LBFS~\cite{muthitacharoen2001low} exploits similarities between files
or versions of the same file. Over the past two years, we have seen a plethora
of geo-distributed analytical frameworks:
WANalytics~\cite{vulimiri2015wananlytics}, Geode~\cite{vulimiri2015global},
Iridium~\cite{pu2015low}, Pixida~\cite{kloudas2015pixida},
Clarinet~\cite{viswanathan2016clarinet}, etc. They minimize data movement by
incorporating heterogeneous wide-area bandwidth information into query
optimization and data/task placement. While effective in speeding up analytics,
these systems focus on batch tasks such as MapReduce jobs or SQL queries. Such
tasks are often executed once, without real time constrain. In contrast,
\sysname{} processes streams continuously in real time.

%% - Pixida~\cite{kloudas2015pixida} minimizes data movement across inter-DC
%% links by solving a traffic minimization problem in data analytics.

%% - Iridium~\cite{pu2015low} optimizes data and task placement jointly to
%% achieve an overall low latency goal.

%% - Clarinet~\cite{viswanathan2016clarinet} incorporate bandwidth information
%% into query optimizer to reduce data transfer.

%% WheelFS~\cite{stribling2009flexible}

%% Geode~\cite{vulimiri2015global} develops input data movement and join
%% algorithm selection strategies to minimize WAN bandwidth usage.

%% WANalytics~\cite{vulimiri2015wananlytics} focuses on batch analytics.

%% SWAG~\cite{hung2015scheduling} coordinates compute task scheduling across
%% DCs.

%% \cite{heintz2015towards} discusses the complex tradeoffs involved in
%% wide-area analytics.

\para{(Adaptive) video streaming.} Multimedia streaming protocols, such as
RTP~\cite{schulzrinne2006rtp}, aims to be fast instead of reliable. While they
can achieve low latency, their accuracy can be poor without adaptation.  Recent
works on multimedia streaming have moved towards using TCP (primarily
HTTP-based) and focused on designing good adaptation strategy to improve QoE, as
in research projects~\cite{mao2017neural, sun2016cs2p, yin2015control} and
industrial efforts (HLS~\cite{pantos2016http}, DASH~\cite{sodagar2011mpeg,
  michalos2012dynamic}). These adaptation strategies work great for video on
demand (VoD), but they are not for ultra low latency streaming: $(i)$ they use
pulling: the client keeps refreshing index file and fetching new chunks; $(ii)$
chunking introduces latency: each chunk has to be ready before the client
fetches.  In addition, because they optimize QoE for humans, the adaptation can
be a mismatch for analytics that rely on image details (e.g.\,10\% accuracy for
PD). In contrast, \sysname{} learns adaptation strategy for each application
(also not limited to video analytics); and the runtime is optimized for ultra
low latency.

\para{QoS.} Most QoS works~\cite{ferrari1990scheme, shenker1994integrated,
  shenker1995fundamental} in the 1990s focus on network-layer solutions that are
not widely deployable. \sysname{} adopts an end-host and application-layer
approach ready today for WAN. For existing application-layer
approaches~\cite{vandalore2001survey}, \sysname{}'s APIs can incorporate their
techniques, such as encoding the number of layers as a knob to realize the
layered approach in Rejaie et al~\cite{rejaie2000layered}. Fundamentally,
\sysname{} does not provide hard QoS guarantees; instead \sysname{} maximizes
achievable accuracy (application performance) with respect to bandwidth
constraints, such that latency is minimized (system performance): a
multidimensional optimization.

%% TCP, for example, adapts to available bandwidth by controlling the congestion
%% window with AIMD~\cite{jacobson1988congestion}.  Previous work has proposed
%% modifications to TCP for specific contexts. For streaming over TCP, Goel et
%% al.~\cite{goel2008low} proposes adaptive buffer-size tuning. For the cloud,
%% Adaptive TCP~\cite{wu2013adaptive} proposes to modify the congestion control
%% behavior based on flow size for the cloud.

% \para{Scheduling and allocation:} Resource allocations in the cloud is how to
% efficiently allocate tasks.  In the wide area, we face fundamental limit that
% therefore degradation is more effective. For multiple tasks, Existing
% allocations are for resources without considering application trade-offs.
% MediaNet~\cite{hicks2003user} uses both local and online global resource
% scheduling to improve user performance and network utilization, and adapts
% without requiring underlying support for resource
% reservations. VideoStorm~\cite{zhang2017live} primarily focuses on cluster
% resource allocation among video queries. For wide-area, the resource
% allocation is limited. We do not control the capacity; but still we can
% allocate the available bandwidth.

% Performance modeling: CherryPick~\cite{alipourfard2017cherrypick},
% Ernest~\cite{venkataraman2016ernest}.
%% Dolly~\cite{ananthanarayanan2013effective}.
%% \para{Program Synthesis:}

%%% Local Variables:
%%% mode: latex
%%% TeX-master: "awstream"
%%% End:
