\section{Implementation}
\label{sec:implementation}

While our proposed API is general and not language specific, we choose a safe
language, Rust, to implement our prototype. \sysname{} is open-source on
Github.\footnote{URL elided for anonymity.}  Applications built with \sysname{}
run as a single process.  The execution mode---profiling, runtime as client, or
runtime as server---is configurable with command line arguments or environment
variables. We implemented the deployment manager that can start, stop and update
the modules in clients and servers.

\begin{table}
  \scriptsize
  \centering
  \begin{tabular}{c c c c}
    \toprule
    Application & Knobs & Accuracy & Dataset \\
    \midrule
    \specialcell{Augmented\\Reality}
                & \specialcell{resolution \\ frame rate \\ quantization }
                & F1 score~\cite{Rijsbergen:1979:IR:539927}
                & \specialcell{iPhone video clips\\training: office (24
    s)\\testing: home (246 s)} \\
    \midrule
    \specialcell{Pedestrian\\Detection}
                & \specialcell{resolution \\ frame rate \\ quantization }
                & F1 score
                & \specialcell{MOT16~\cite{milan2016mot16}\\training: MOT16-04\\testing: MOT16-03} \\
    \midrule
    \specialcell{Log Analysis\\(Top-K, K=50)}
                & \specialcell{head (N) \\ threshold (T) }
                & \specialcell{Kendall's $\tau$~\cite{abdi2007kendall}}
                & \specialcell{\href{https://www.sec.gov}{SEC.gov} logs~\cite{edgarlog} \\ training: 4 days \\
    testing: 16 days} \\
    \bottomrule
  \end{tabular}
  \caption{\sysname{} Applications}
  \label{tab:apps}
\end{table}

We've built three applications: augmented reality (AR), pedestrian detection
(PD), and a distributed log analysis to extract the Top-K mostly accessed
files. \autoref{tab:apps} summarizes the application-specific part: knobs,
accuracy function, and the dataset we used for training and testing. To conserve
space, the appendix describes applications in detail.

%%% Local Variables:
%%% mode: latex
%%% TeX-master: "../awstream"
%%% End:
