\section{Implementation}
\label{sec:implementation}

While our proposed API is general and not language specific, we have implemented
\sysname{} prototype in Rust (\textasciitilde 4000 lines of code). \sysname{} is
open source on GitHub.\footnote{URL elided for anonymity.}  Applications use
\sysname{} as a library and configure the execution mode---profiling, runtime as
client, or runtime as server---with command line arguments.  We have implemented
the deployment manager that can start, stop, and update clients/servers as shell
scripts.

\begin{table}
  \small
  \centering
  \begin{tabular}{c c c c}
    \toprule
    Application & Knobs & Accuracy & Dataset \\
    \midrule
    \specialcell{Augmented\\Reality}
                & \specialcell{resolution \\ frame rate \\ quantization }
                & F1 score~\cite{Rijsbergen:1979:IR:539927}
                & \specialcell{iPhone video clips\\training: office (24
    s)\\testing: home (246 s)} \\
    \midrule
    \specialcell{Pedestrian\\Detection}
                & \specialcell{resolution \\ frame rate \\ quantization }
                & F1 score
                & \specialcell{MOT16~\cite{milan2016mot16}\\training: MOT16-04\\testing: MOT16-03} \\
    \midrule
    \specialcell{Log Analysis\\(Top-K, K=50)}
                & \specialcell{head (N) \\ threshold (T) }
                & \specialcell{Kendall's $\tau$~\cite{abdi2007kendall}}
                & \specialcell{\href{https://www.sec.gov}{SEC.gov} logs~\cite{edgarlog} \\ training: 4 days \\
    testing: 16 days} \\
    \bottomrule
  \end{tabular}
  \caption{\sysname{} Applications}
  \label{tab:apps}
  \vspace{-1em}
\end{table}

Using \sysname{}, we have built three applications: augmented reality (AR) that
recognizes nearby objects on mobile phones, pedestrian detection (PD) for
surveillance cameras, and a distributed log analysis to extract the Top-K mostly
accessed files (TK). \autoref{tab:apps} summarizes the application-specific
part: knobs, accuracy function, and the dataset we used for training and
testing. Below we provide details about how we implement our three applications
(\autoref{fig:three-apps}), including the libraries we use and how we integrate
them into \sysname{}.

\para{Augmented Reality.} We target at mobile augmented reality applications
that recognize objects by offloading the heavy computation to resources
elsewhere, e.g.\,the cloud.  Image-related operations use OpenCV
3.1~\cite{opencvlibrary} and object recognition uses YOLO~\cite{darknet13,
  redmon2016yolo9000}, a GPU-enabled pre-trained neural network. Videos are
encoded with H.264~\cite{richardson2011h} because of its prevalence in existing
systems. Our implementation uses GStreamer~\cite{gstreamer} with
\texttt{x264enc} plugin. To integrate with \sysname{}, we first create a
pipeline that exposes \texttt{appsrc} (to feed raw image data) and
\texttt{appsink} (to get encoded bytes). The GStreamer main loop executes in a
separate thread and \sysname{} communicates with it via Rust's channel. The
\texttt{x264enc} uses the \texttt{zerolatency} preset and four threads. We use
constant quality encoding and expose the quantization factor as a knob (in
addition to image resolutions and frame rates).

Object recognition returns a list of bounding boxes with the type of the object,
and each bounding box is a rectangle with normalized coordinates on the
image. We compare the detection against the reference result from raw data, and
declare it success if the intersection over union (IOU) is greater than
50\%~\cite{everingham2010pascal} and the object type matches. We use F1
score~\cite{Rijsbergen:1979:IR:539927} as the accuracy function. In terms of
dataset, we collected our own video clips: the training data is a 24-second long
video of an office environment; the test data is a 246-second long video of a
home environment.

\para{Pedestrian Detection.} This application analyzes streams of videos from
installed CCTV cameras and detects pedestrians inside. We use a similar setup
(OpenCV and GStreamer) as our augmented reality application except for the
analytical function. To detect pedestrians, we use histogram of oriented
gradients (HOG)~\cite{dalal2005histograms} with the default linear SVM
classifier from OpenCV. To ensure real-time processing of frames,
GPU-accelerated implementation is used. Because we do not recognize individual
pedestrians, a successful detection in this case only requires matching the
bounding box. Our evaluation uses MOT16 dataset~\cite{milan2016mot16} for both
profiling and runtime.

\begin{figure}
  \centering
  \includegraphics[width=\columnwidth]{figures/apps.pdf}
  \caption{Three applications: augmented reality, pedestrian detection, and
    distributed Top-K.}
  \label{fig:three-apps}
\end{figure}

\begin{figure}
  \centering
  \includegraphics[width=\columnwidth]{figures/topk.pdf}
  \caption{A distributed Top-K application with two degradation operations:
    \texttt{head} and \texttt{threshold}. Discarding data with these two
    operations will affect final results. In this example, \texttt{f2}, which is
    not in Top-1 for either client, becomes the global Top-1 after the merge. It
    would have been purged if the clients use threshold T=3.}
  \label{fig:topk}
\end{figure}

\para{Distributed Top-K.} Many monitoring applications need to answer the
\textit{Top-K} question~\cite{babcock2003distributed}, such as the Top-K most
popular URLs, or the Top-K most access files. A distributed Top-K application
aggregates information from geo-distributed servers to computer a final Top-K.

\autoref{fig:topk} shows the Top-K processing pipeline with example data. Source
nodes first performs a \texttt{Window} operation to generate data summary, which
is key-value pairs \texttt{(item: count)} for each item. After the summary, the
data size can still be too large because most real-world access patterns follow
a long tail distribution: there is a large-but-irrelevant tail that contributes
little to Top-K. The source nodes perform two degradation operations: (1) a
head(\texttt{N}) operation that only takes the top \texttt{N} entries; (2) a
threshold(\texttt{T}) that filters small entries whose count is smaller than
\texttt{T}. These two operations are not orthogonal. Their impact on data size
reduction and quality degradation depends on the data distribution.

For accuracy, we use Kendall's~$\tau$~\cite{abdi2007kendall}, a correlation
measure of the concordance between two ranked list. The output ranges from
\(-1\) to 1, representing no agreement to complete agreement. To integrate with
\sysname{}, we convert Kendall's~$\tau$ to the range of [0, 1] with a linear
transformation.

Our Top-K application aims to find the Top-50 most accessed files from web
server logs. We use Apache log files that record and store user access
statistics for the \href{https://www.sec.gov}{SEC.gov} website. The logs are
split into four groups, simulating four geo-distributed nodes monitoring web
accesses. To match the load of popular web servers, we compress one hour's logs
into one second.

%%% Local Variables:
%%% mode: latex
%%% TeX-master: "../awstream"
%%% End:
