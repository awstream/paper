\subsection{Automatic Profiling}
\label{sec:automatic-profiling}

The goal of the profiling is to learn how different levels of degradation
affects the bandwidth demand and application accuracy. By exploring this
trade-off space, \sysname{} generates a Pareto-optimal \textit{profile} for each
application. We formulate the profiling and discuss the designs for offline and
online profiling.

\para{Profiling formalism:} Suppose in a stream processing application that $n$
\maybe{} operators are used. Each introduces a knob $k_i$. We assume these
operators are independent from each other; their combination forms a
\textit{configuration} $c = [k_{1}, k_{2}, ... k_{n}]$. The set of all possible
configurations $\mathbb{C}$ is the space that our profiling system needs to
explore. For each configuration $c$, there are two mappings that our system
needs to explore: a mapping from $c$ to its bandwidth requirement $B(c)$ and its
accuracy measure $A(c)$. \autoref{tab:notations} summarizes the symbols used in
this paper.

The Pareto-optimal set $\mathbb{P}$ is then defined (\autoref{eq:pareto}): for
all $c \in \mathbb{P}$, there is no alternative configuration $c'$ that requires
less bandwidth while giving a higher accuracy.

{\small
\begin{equation}
  \mathbb{P} = \{ c \in \mathbb{C} : \{ c' \in \mathbb{C}: B(c') < B(c),
  A(c') > A(c) \} = \varnothing\}
  \label{eq:pareto}
\end{equation}
}%

\begin{table}
  \footnotesize
  \centering
  \begin{tabular}{r l}
    \toprule
    \textbf{Symbol} & \textbf{Description} \\
    \midrule
    $n$ & number of degradation operations \\
    $k_i$ & the \textit{i}-th degradation knob \\
    $c = [k_{1}, k_{2}, ... k_{n}]$ & one specific configuration \\
    $\mathbb{C}$ & the set of all configurations \\
    \midrule
    $B(c)$ & bandwidth requirement for $c$ \\
    $A(c)$ & accuracy measure for $c$ \\
    $\mathbb{P}$ & Pareto efficienct set \\
    \bottomrule
  \end{tabular}
  \caption{Notations used in this paper.}
  \label{tab:notations}
\end{table}

Since there is often no closed form relation for $B(c)$ and $A(c)$, our system
takes a data-driven approach: with a representative training dataset and an
application-specific utility function, our system evaluates each configuration
for its bandwidth and accuracy. The accuracy can either be measured against the
groundtruth; or in the case when labelled dataset is not available, \sysname{}
uses the results when all degradations are turned off as the reference. We will
discuss concrete knobs, configurations, $B(c)$ and $A(c)$ when we present the
example applications in \autoref{sec:build-appl}.

\para{Offline Profiling:} At the level of being a general API, our profiling
requires an exhaustive search. Developers using \sysname{} will first perform an
offline profiling that generates the Pareto-optimal profile by supplying
training dataset and application accuracy function. Because our APIs are
general, there is no restriction in the values or functions provided for
\maybe{}. Without \textit{a prior}, any configuration is possible to be inside
the optimal profile.

The exhaustive search is expensive. Therefore, the offline profiling requires to
explore the entire parameter space, which is a combinatorial space of all
knobs. For \textit{offline}, the time required is acceptable. We note that
parallelism is usable for profiling. Except for the groundtruth (or reference
label), two different configurations have no dependence over each other. Making
the profiling an embarrassingly parallel task.

If degradation operations' impact is can be predicted with a prediction
framework (such as Ernest~\cite{venkataraman2016ernest}), the profiler can
estimate a subset of the configuration space.

\para{Online Profiling:} The offline profile is susceptible to ``model
drift''. When the required bandwidth is underestimated, data will be queued at
the sender and application latency will increase. When the accuracy is
incorrectly measured, the actual application performance will be suboptimal. The
drift needs to be corrected online. There are two particular challenges for
online profiling.

The first is the lack of groundtruth data or reference data. During the online
execution, it's often infeasible to get labelled data as the groundtruth. For
example, image labelling is known to be labor intensive and time
consuming~\cite{russell2008labelme}. In addition, the original un-degraded data
need to be transmitted from the sender. \sysname{} addresses labelling by using
the un-degraded data as the reference and allocate additional bandwidth for
backhaulling portions of the un-degraded data. While the additional bandwidth
seems a waste, in our design, during the runtime's probing phase, the original
data can enjoy a free ride. Details of the probing is in \autoref{sec:runtime}.

The second challenge is how to profile efficiently. We use the offline profiling
information as a prior to speed-up the online profiler. Specficially, we employ
two techniques: degradation-aware parallelization and partial profiling.

\para{Degradation-aware parallelization:} The profiling tasks of exploring all
configurations are easily parallelizable. Normal job schedulers don't assume the
knowledge of estimated task completion time, therefore the parallel execution
suffers from sub-optimal assignments. Aided with the offline profiling
statistics (the time it requires to profile a particular configuration), we can
schedule the online profiling tasks with a longest first scheduling that
minimizes the makespan.

\para{Partial profiling:} In time domain, the profiling could use a smaller
chunk of data to approximate the best strategy. Besides, we can profiling a
subset of the total configurations and measures its difference from the current
profile. If the difference exceeds a certain threshold, it triggers a full
profiling.

%%% Local Variables:
%%% mode: latex
%%% TeX-master: "sosp17"
%%% End:
