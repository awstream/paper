\subsection{Automatic Profiling}
\label{sec:automatic-profiling}

After developers use \maybe{} operators to specify potential degradation operations,
\sysname{} automatically builds an accurate application profile. The profile
captures the relationship between application accuracy and bandwidth consumption
under different combinations of data degradation operations. We describe the
formalism, followed by techniques that efficiently perform offline and online
profiling.

\para{Profiling formalism.} Suppose a stream processing application has $n$
\maybe{} operators. Each operator introduces a knob $k_i$ and their combination
forms a \textit{configuration} $c = [k_{1}, k_{2}, ... k_{n}]$. The set of all
possible configurations $\mathbb{C}$ is the space that the profiling
explores. For each configuration $c$, there are two mappings that are of
particular interest: a mapping from $c$ to its bandwidth consumption $B(c)$ and
its accuracy measure $A(c)$. \autoref{tab:notations} summarizes the notations used in
this paper.

The profiling looks for all $c$ such that there is no alternative
configuration $c'$ that requires less bandwidth while giving a higher
accuracy. These configurations form the Pareto-optimal set $\mathbb{P}$, defined
as follows:

{\small \vspace{-1em}
  \begin{equation}
  \mathbb{P} = \{ c \in \mathbb{C} : \{ c' \in \mathbb{C}: B(c') < B(c),
  A(c') > A(c) \} = \varnothing\}
  \label{eq:pareto}
\end{equation}
}%

\begin{table}
  \footnotesize
  \centering
  \begin{tabular}{r l}
    \toprule
    \textbf{Symbol} & \textbf{Description} \\
    \midrule
    $n$ & number of degradation operations \\
    $k_i$ & the \textit{i}-th degradation knob \\
    $c = [k_{1}, k_{2}, ... k_{n}]$ & one specific configuration \\
    $\mathbb{C}$ & the set of all configurations \\
    \midrule
    $B(c)$ & bandwidth requirement for $c$ \\
    $A(c)$ & accuracy measure for $c$ \\
    $\mathbb{P}$ & Pareto-optimal set \\
    \midrule
    $c_i$, $c_{i+1}$, $c_{\max}$ & current/next/maximal configuration at runtime \\
    $R$ & network delivery rate (estimated bandwidth) \\
    $\text{Q}_\text{E}$, $\text{Q}_\text{C}$ & messages when \texttt{Queue} is empty or congested \\
    $\text{S}_\text{ProbeDone}$ & message when \texttt{Socket} finishes probing \\
    \bottomrule
  \end{tabular}
  \caption{Notations used in this paper.}
  \label{tab:notations}
\end{table}

Because \sysname{} allows arbitrary functions as the degradation functions, it
does not assume a closed-form relation for $B(c)$ and $A(c)$. Instead,
\sysname{} takes a data-driven approach: profiling applications with
developer-supplied training data.  We measure $B(c)$ at the point of
transmission. The accuracy $A(c)$ is measured either against the ground-truth,
or the reference results when all degradation operations are off.  We discuss
examples of concrete knobs, configurations, accuracy functions when we present
applications in \autoref{sec:implementation}.

\para{Offline Profiling.} We first use an offline process to build a bootstrap
profile (or default profile).
\sysname{} makes no assumptions on the performance models, and thus evaluate all possible configurations.
While all knobs form a combinatorial space, the offline profiling is only a one-time process.
We exploits parallelism to reduce the profiling time.
Without any \textit{a priori} knowledge, all
configurations are assigned randomly to available machines.
Nevertheless, given certain workloads, we can leverage statistical methods to build accurate performance
models while only exploring a small number of configurations~\cite{venkataraman2016ernest,
  alipourfard2017cherrypick}.

%\para{Offline Profiling.} We first use an offline process to build a bootstrap
%profile (or default profile).
%Because \sysname{} supports arbitrary degradation operations, we need to evaluate all combinations of the configurations
%offline profiling is a one-time process,
%\sysname{} currently performs an exhaustive evaluation of all configurations in
%$\mathbb{C}$ despite all knobs form a combinatorial space. Future work could
%explore statistical methods to build performance models with a smaller number of
%training samples~\cite{venkataraman2016ernest, alipourfard2017cherrypick}.
%\sysname{} exploits parallelism when profiling all configurations.
%Without any \textit{a priori} knowledge, all
%configurations are assigned randomly to all available machines.

\para{Online Profiling:} \sysname{} runs an online profiling process
continuously to refine the profile. The refinement handles \textit{model
  drifts}, a problem when the learned profile fails to predict the performance
accurately. There are two challenges with online profiling.
$(i)$~There is no ground-truth labels or reference data to compute
accuracy. Because labelling data is prohibitively labor intensive and time
consuming~\cite{russell2008labelme}, \sysname{} currently uses raw data as the
reference. If the application is run without any data degradation, the
application data is used for online profiling. Otherwise, we allocate additional
bandwidth to back-haul raw data, but only do so when it does not interfere the
running application.
$(ii)$~Exhaustive profiling is expensive. If it takes too much time,
the newly-learned profile may already be stale. \sysname{} uses a combination of
parallelism and sampling techniques to speed up this process, as described below:

\begin{itemize}[leftmargin=*]

\item Degradation-aware parallelization: Evaluating each configuration takes a
different amount of time. Typically, an increase in the level of degradation
leads to a decrease in computation, e.g.\,the smaller the FPS, the fewer images
to process. Therefore, we collect processing times for each configuration from
the offline process and use them for scheduling---for example longest-job first
schedule (LFS)~\cite{karger2010scheduling}---in parallelization.

\item Sampling-based profiling: Online profiling can speed up if we only profile a subset of all data or a subset of all configurations. 
The former case reduces the amount of data to process, but at a cost of generating a less accurate profile. In the latter case, we can evaluate a subset of the configurations and compare their performances with the existing profile. A substantial difference, such as more than 1 mbps of bandwidth estimation, triggers a full profiling over all configurations to update the current profile.

\end{itemize}

%%% Local Variables:
%%% mode: latex
%%% TeX-master: "sosp17"
%%% End:
