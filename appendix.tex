\documentclass[twocolumn, 9pt]{article}

\usepackage{usenix}

\usepackage[T1]{fontenc}
\usepackage[explicit]{titlesec}
\usepackage[font=small,labelfont=bf]{caption}
% \usepackage[toc,page]{appendix}
\usepackage{algpseudocode}
\usepackage{amsmath}
\usepackage{amssymb}
\usepackage{balance}
\usepackage{booktabs} % For formal tables
\usepackage{enumitem}
\usepackage{hyperref}
\usepackage{listings}
\usepackage{microtype}
\usepackage{multirow}
\usepackage{subcaption}
\usepackage{tikz}
\usepackage{varwidth}
\usepackage{xcolor}

\hypersetup{
  colorlinks=true,
  linkcolor=ACMRed,
  urlcolor=ACMBlue,
  citecolor=ACMRed,
}

\definecolor[named]{ACMBlue}{cmyk}{1,0.1,0,0.1}
\definecolor[named]{ACMYellow}{cmyk}{0,0.16,1,0}
\definecolor[named]{ACMOrange}{cmyk}{0,0.42,1,0.01}
\definecolor[named]{ACMRed}{cmyk}{0,0.90,0.86,0}
\definecolor[named]{ACMLightBlue}{cmyk}{0.49,0.01,0,0}
\definecolor[named]{ACMGreen}{cmyk}{0.20,0,1,0.19}
\definecolor[named]{ACMPurple}{cmyk}{0.55,1,0,0.15}
\definecolor[named]{ACMDarkBlue}{cmyk}{1,0.58,0,0.21}

\lstset{
  captionpos=b,
  showspaces=false,
  showtabs=false,
  breaklines=true,
  showstringspaces=false,
  breakatwhitespace=true,
  escapeinside={(*@}{@*)},
  basicstyle=\footnotesize\ttfamily,
  columns=fullflexible,
  morekeywords={maybe_downsample, maybe_skip, maybe}
}

\captionsetup[lstlisting]{font={footnotesize}}
\renewcommand{\lstlistingname}{Example}

% \makeatletter
% \newcommand\appendix@section[1]{%
%   \refstepcounter{section}%
%   \orig@section*{Appendix \@Alph\c@section: #1}%
%   \addcontentsline{toc}{section}{Appendix \@Alph\c@section: #1}%
% }
% \let\orig@section\section
% \g@addto@macro\appendix{\let\section\appendix@section}
% \makeatother

% \newcommand*{\Appendixautorefname}{Appendix}

\frenchspacing

%%%%%%%%%%%%%%%%%%%%%%%%%%%%%%%%%%%%%%%%%%%%%%%%%%%%%%%%%%%%%%%
%% Paper setup
%%%%%%%%%%%%%%%%%%%%%%%%%%%%%%%%%%%%%%%%%%%%%%%%%%%%%%%%%%%%%%%

\newcommand{\sysname}{AWStream}
\newcommand{\para}[1]{\smallskip\noindent\textbf{#1}}
\newcommand{\paraf}[1]{\noindent\textbf{#1}}
\newcommand{\todo}[1]{{\color{ACMRed}\bf{TODO: #1}\normalfont}}
\newcommand{\fixme}[1]{{\color{ACMRed}\bf{FIXME: #1}\normalfont}}
\newcommand{\question}[1]{{\color{ACMRed}\footnotesize{Q: #1}\normalfont}}

\newcommand{\maybe}{\texttt{maybe}}

\def\Snospace~{\S{}}
\renewcommand*\sectionautorefname{\Snospace}
\renewcommand*\subsectionautorefname{\Snospace}
\renewcommand*\subsubsectionautorefname{\Snospace}
\renewcommand*{\equationautorefname}{Eq.}
\renewcommand*{\figureautorefname}{Fig.}

\newcommand{\specialcell}[2][c]{%
  \begin{tabular}[#1]{@{}c@{}}#2\end{tabular}}

\newcommand*\circled[1]{\tikz[baseline=(char.base)]{
    \node[shape=circle,draw,inner sep=0.5pt] (char) {#1};}}

\newcommand*\qe{$\text{Q}_\text{E}$}
\newcommand*\qc{$\text{Q}_\text{C}$}
\newcommand*\rc{$\text{R}_\text{C}$}
\newcommand*\spd{$\text{S}_\text{ProbeDone}$}

\begin{document}
\title{\sysname{}: Adaptive Wide-Area Streaming Analytics \\ Appendix}
\author{ \textit{Paper \#8} }
\date{}
\maketitle

\section{Application Implementations}
\label{appendix:appl-impl}

Below we provide details about how we implement our three applications
(\autoref{fig:three-apps}), including the libraries we use and how we integrate
them into \sysname{}.

\para{Augmented Reality.} We target at mobile augmented reality applications
that recognizes objects by offloading the heavy computation to resources
elsewhere, e.g.\,the cloud.  Image-related operations use OpenCV
3.1~\cite{opencvlibrary} and object recognition uses YOLO~\cite{darknet13,
  redmon2016yolo9000}, a GPU-enabled pre-trained neural network. Videos are
encoded with H.264~\cite{richardson2011h} because of its prevalence in existing
systems. Our implementation uses GStreamer~\cite{gstreamer} with
\texttt{x264enc} plugin. To integrate with \sysname{}, we first create a
pipeline that exposes \texttt{appsrc} (to feed raw image data) and
\texttt{appsink} (to get encoded bytes). The GStreamer main loop executes in a
separate thread and \sysname{} communicates with it via Rust's channel. The
\texttt{x264enc} uses the \texttt{zerolatency} preset and four threads. We use
constant quality encoding and expose the quantization factor as a knob (in
addition to image resolutions and frame rates).

Object recognition returns a list of bounding boxes with the type of the object,
and each bounding box is a rectangle with normalized coordinates on the
image. We compare the detection against the reference result from raw data, and
declare it success if the intersection over union (IOU) is greater than
50\%~\cite{everingham2010pascal} and the object type matches. We use F1
score~\cite{Rijsbergen:1979:IR:539927} as the accuracy function. In terms of
dataset, we collected our own video clips: the training data is a 24-second long
video of an office environment; the test data is a 246-second long video of a
home environment.

\para{Pedestrian Detection.} This application analyzes streams of videos from
installed CCTV cameras and detects pedestrians inside. We use a similar setup
(OpenCV and GStreamer) as our augmented reality application except for the
analytical function. To detect pedestrians, we use histogram of oriented
gradients (HOG)~\cite{dalal2005histograms} with the default linear SVM
classifier from OpenCV. To ensure real-time processing of frames,
GPU-accelerated implementation is used. Because we do not recognize individual
pedestrians, a successful detection in this case only requires matching the
bounding box. Our evaluation uses MOT16 dataset~\cite{milan2016mot16} for both
profiling and runtime.

\begin{figure}
  \centering
  \includegraphics[width=\columnwidth]{figures/apps.pdf}
  \caption{Three applications: augmented reality, pedestrian detection, and
    distributed Top-K.}
  \label{fig:three-apps}
\end{figure}

\begin{figure}
  \centering
  \includegraphics[width=\columnwidth]{figures/topk.pdf}
  \caption{A distributed Top-K application with two degradation operations:
    \texttt{head} and \texttt{threshold}. Discarding data with these two
    operations will affect final results. In this example, \texttt{f2}, which is
    not in Top-1 for either client, becomes the global Top-1 after the merge. It
    would have been purged if the clients use threshold T=3.}
  \label{fig:topk}
\end{figure}

\para{Distributed Top-K.} Many monitoring applications need to answer the
\textit{Top-K} question~\cite{babcock2003distributed}, such as the Top-K most
popular URLs, or the Top-K most access files. A distributed Top-K application
aggregates information from geo-distributed servers to computer a final Top-K.

\autoref{fig:topk} shows the Top-K processing pipeline with example data. Source
nodes first performs a \texttt{Window} operation to generate data summary, which
is key-value pairs \texttt{(item: count)} for each item. After the summary, the
data size can still be too large because most real-world access patterns follow
a long tail distribution: there is a large-but-irrelevant tail that contributes
little to Top-K. The source nodes perform two degradation operations: (1) a
head(\texttt{N}) operation that only takes the top \texttt{N} entries; (2) a
threshold(\texttt{T}) that filters small entries whose count is smaller than
\texttt{T}. These two operations are not orthogonal. Their impact on data size
reduction and quality degradation depends on the data distribution.

For accuracy, we use Kendall's~$\tau$~\cite{abdi2007kendall}, a correlation
measure of the concordance between two ranked list. The output ranges from
\(-1\) to 1, representing no agreement to complete agreement. To integrate with
\sysname{}, we convert Kendall's~$\tau$ to the range of [0, 1] with a linear
transformation.

Our Top-K application aims to find the Top-50 most accessed files from web
server logs. We use Apache log files that record and store user access
statistics for the \href{https://www.sec.gov}{SEC.gov} website. The logs are
split into four groups, simulating four geo-distributed nodes monitoring web
accesses. To match the load of popular web servers, we compress one hour's logs
into one second.

\section{Runtime Evaluation of PD and TK}
\label{appendix:more-runtime}

\begin{figure}[t]
  \begin{subfigure}[t]{\columnwidth}
    \centering
    \includegraphics[width=\columnwidth]{figures/runtime_mot-timeseries.pdf}
    \caption{PD's runtime behavior with a time-series plot: throughput (top),
      showing the effect of bandwidth shaping; latency (middle) in log scale;
      and accuracy (bottom).}
    \label{fig:pd-runtime-timeseries}
  \end{subfigure}
  \vspace{1em}
  \\
  \begin{subfigure}[t]{\columnwidth}
    \centering
    \includegraphics[width=\columnwidth]{figures/runtime_mot-boxplot.pdf}
    \caption{PD's performance summary of latency and accuracy during the traffic
      shaping (between t=200s and t=440s).}
\label{fig:pd-runtime-boxplot}
  \end{subfigure}
  \caption{PD runtime evaluation.}
  \label{fig:pd-runtime}
\end{figure}

\begin{figure}[t]
  \begin{subfigure}[t]{\columnwidth}
    \centering
    \includegraphics[width=\columnwidth]{figures/runtime_tk-timeseries.pdf}
    \caption{TK's runtime behavior with a time-series plot: throughput (top),
      showing the effect of bandwidth shaping; latency (middle) in log scale;
      and accuracy (bottom).}
    \label{fig:tk-runtime-timeseries}
  \end{subfigure}
  \vspace{1em}
  \\
  \begin{subfigure}[t]{\columnwidth}
    \centering
    \includegraphics[width=\columnwidth]{figures/runtime_tk-boxplot.pdf}
    \caption{TK's performance summary of latency and accuracy during the traffic
      shaping (between t=200s and t=440s).}
    \label{fig:tk-runtime-boxplot}
  \end{subfigure}
  \caption{TK runtime evaluation.}
  \label{fig:tk-runtime}
\end{figure}

\begin{figure*}[!htb]
  \centering
  \includegraphics[width=\textwidth]{figures/hls-arch.pdf}
  \caption{HLS setup. (Left) High-level overview: machine 1 generates video and
    stores it in the server; machine 2 fetches the data and performs
    analytics. (Right) Detailed view: (1)~FFmpeg encodes video with multiple
    bitrates and groups them into MPEG-TS programs; (2)~\texttt{nginx-ts-module}
    then generates HLS chunks on the fly and store them for nginx serving;
    (3)~the client (using \texttt{hls.js}) periodically fetches the latest index
    file (\texttt{index.m3u8}) and then download right chunk according to
    network conditions.}
  \label{fig:hls-arch}
\end{figure*}

\paraf{Pedestrian Detection.} The setup for PD is the same with AR: three Amazon
EC2 as clients and one as server. The maximal configuration $c_{\max}$ is
1920x1080 resolution, \(10~\text{FPS}\) and a quantization of 20; it consumes
about 12 mbps. When running experiments, we use the same bandwidth shaping
schedule. Baselines are also the same: Streaming over TCP/UDP, JetStream, and
JetStream++.

\autoref{fig:pd-runtime} shows the runtime behavior of \sysname{} and all
baselines in time series (throughout the experiment) and box plot (between
t=200s and t=440s). The trend is quite similar to AR, so we omit a verbose
description.

\para{Top-K.} For TK, with our data is split into four groups, we use four
Amazon EC2 as clients and one as server. We also limit the maximal configuration
$c_{\max}$ as $N=9750$ for \texttt{head} and $T=0$ for \texttt{threshold}; it
consumes about 1.2 mbps. Because the overall bandwidth consumption is much
smaller than video analytics, we change the bandwidth parameter: during
t=200-380s, we limit the bandwidth to 750 kbps; during t=380-440s, the bandwidth
is 500 kbps. The background bandwidth limit is 2.5 mpbs.

We've only compared \sysname{} with two baselines: streaming over TCP and
streaming over UDP. For TCP baseline, we use AWStream but disable the
adaptation. For UDP baseline, we implemented a simple application
protocol---packetization and a custom header---so that the receiver can still
aggregate data in the presence of UDP packet loss.

\autoref{fig:tk-runtime} shows the evaluation results for Top-K
application. Streaming over TCP has the highest accuracy (\textasciitilde
99.7\%) but the worst latency (up to 40 seconds). Streaming over UDP has the low
latency but its accuracy is the worst (\textasciitilde 52\%). \sysname{}
achieves low latency (\textasciitilde 1.1 second) and high accuracy
(\textasciitilde 98\%) simultaneously. Notice that because TK's source generates
data every second after the window operation, one object in the queue leads to
one seconds latency. \sysname{} manages to avoid queues for most times.

% > summary(latency)
%       Time       Streaming over TCP Streaming over UDP    AWStream
%  Min.   :210.0   Min.   : 2036      Min.   :22.29      Min.   :  48.03
%  1st Qu.:251.2   1st Qu.:11438      1st Qu.:22.42      1st Qu.: 485.08
%  Median :292.5   Median :21014      Median :24.78      Median : 946.45
%  Mean   :292.5   Mean   :20590      Mean   :23.85      Mean   :1145.05
%  3rd Qu.:333.8   3rd Qu.:29662      3rd Qu.:24.87      3rd Qu.:1557.20
%  Max.   :375.0   Max.   :39434      Max.   :24.98      Max.   :3509.99
% > summary(accuracy)
%       Time       Streaming over TCP Streaming over UDP    AWStream
%  Min.   :210.0   Min.   :0.9808     Min.   :0.1097     Min.   :0.9284
%  1st Qu.:251.2   1st Qu.:0.9892     1st Qu.:0.4467     1st Qu.:0.9694
%  Median :292.5   Median :0.9967     Median :0.5329     Median :0.9800
%  Mean   :292.5   Mean   :0.9928     Mean   :0.5236     Mean   :0.9786
%  3rd Qu.:333.8   3rd Qu.:0.9977     3rd Qu.:0.6063     3rd Qu.:0.9883
%  Max.   :375.0   Max.   :0.9991     Max.   :0.7981     Max.   :0.9991

\section{JetStream++}
\label{appendix:jetstream++}

We modified the open source version of
JetStream\footnote{\url{https://github.com/princeton-sns/jetstream/}, commit
  bf0931b2d74d20fdf891669188feb84c96AF84} in order to use our profile to act as
manual policies. Because JetStream doesn't support simultaneous degradation in
multiple dimensions, we implemented a simple \texttt{VideoSource} operator that
understands how to change image resolutions, frame rate, and video encoding
quantization. At runtime, \texttt{VideoSource} queries congestion policy manager
and adjusts three dimensions simultaneously. This operator is then exposed to
Python-implemented control plane. We call this modified version JetStream++.

JetStream's code base is modular and extensible: the modifications include 53
lines for the header file, 171 lines for implementation, 75 lines for unit test,
and 49 lines of python as the application.

While extending JetStream with our profile is not challenging, JetStream++
performs degradation in a single operator and loses the composability. We could
modify JetStream to support degradation across multiple operators, but that
would require substantial changes to JetStream. Using JetStream++ with our
profile, the comparison is enough to illustrate the difference between
\sysname{} and JetStream's runtime.

Regarding Top-K application, although JetStream provides a Top-K implementation,
it is based on ``Three-Phase Uniform Threshold'' (TPUT) and not suitable for
low-latency Top-K monitoring, according to the original
paper~\cite{cao2004efficient}, \textit{``in our target environments the query is
  asked hourly or daily. The intervals between the queries are typically long
  enough that the top-k objects have changed completely.''} We did not implement
our Top-K degradation with JetStream because video analytics suffices the
purpose of comparison.

%% Patches: https://github.com/nebgnahz/jetstream-clone/pull/2/files

\section{HTTP Live Streaming}
\label{appendix:hls}

In this section, we describe our HLS setup used as the baseline for video
analytics.

\para{Video.} We use \texttt{FFmpeg} encode the video source with
multiple bitrates: all with 30FPS, but with different resolutions and H.264
encoding quality. We then use \texttt{FFmpeg} to re-publish the stream to an
Nginx web server that accepts MPEG-TS over HTTP.

\para{Server.} The server uses nginx with \cite{nginx-ts-module} receives
MPEG-TS over HTTP, produces live HLS and MPEG-DASH chunks. We found that HLS is
more stable than DASH, so in our experiment, we used HLS. We configured each
segment to be 1 second for low latency. Typically 4 seconds~\cite{mao2017neural,
  sun2016cs2p, wang2016anatomy}.

\para{Analytics.} The machine, we did not perform analytics online, instead, we
recorded the chunk information (resolution and encoding quality) and calculate
the application performance (accuracy) offline. Our client uses
Puppeteer~\cite{puppeteer}, Puppeteer is a Node library which provides a
high-level API to control headless Chrome over the DevTools Protocol. It can
also be configured to use full (non-headless) Chrome. and hls.js~\cite{hls.js}
hls.js is a JavaScript library which implements an HTTP Live Streaming
client. It relies on HTML5 video and MediaSource Extensions for playback.

{\footnotesize \bibliographystyle{acm}
  \bibliography{awstream}}

\end{document}
%%% Local Variables:
%%% mode: latex
%%% TeX-master: t
%%% End:
