\subsection{APIs for Structured Adaptation}
\label{sec:struct-adapt}

%% Introduce graphs of operators model
The vast majority of stream processing applications today are constructed as a
directed graph of operators~\cite{toshniwal2014storm, zaharia2013discretized}).
\sysname{} employs such a programming model that integration with existing
applications will require minimal effort. \autoref{tab:operators} lists a few
example operators that \sysname{} supports.

In this model, each operator transforms existing streams into new
streams. Operators are specializable by user-defined functions (UDF). Optionally
developers can provide objects that implement a callable interface. Such objects
can be used to hold application states.

\begin{table*}
  \centering
  \begin{tabular}{ c r l }
    \toprule
    \multirow{7}{*}{Normal Operators}
    & \textit{map} (f: I $\Rightarrow$ O) & Stream<I> $\Rightarrow$ Stream<O> \\
    & \textit{filter} (f: I $\Rightarrow$ bool) & Stream<I> $\Rightarrow$
                                                 Stream<I> \\
    & \textit{skip} n(i: Integer) & Stream<I> $\Rightarrow$ Stream<I> \\
    & \textit{sliding\_window} (count: Integer, f: Vec<I> $\Rightarrow$ O) & Stream<I> $\Rightarrow$
                                                                            Stream<O> \\
    % & \textit{tumbling\_window} (count: Integer, f: Vec<I> $\Rightarrow$ O) & Stream<I> $\Rightarrow$
    %                                                                          Stream<O> \\
    & \textit{timed\_window} (time: Duration, f: Vec<I> $\Rightarrow$ O) & Stream<I> $\Rightarrow$
                                                                          Stream<O> \\
    & ... & ... \\
    \midrule
    \multirow{4}{*}{Degradation Operators}
    & \textit{maybe} (knobs: Vec<T>, f:  (T, I) $\Rightarrow$ I) & Stream<I> $\Rightarrow$
                                                                 Stream<I> \\
    & \textit{maybe\_skip} (knobs: Vec<Integer>) & Stream<I> $\Rightarrow$ Stream<I> \\
    & \textit{maybe\_downsample} (knobs: Vec<(Integer, Interger)>) & Stream<Image> $\Rightarrow$ Stream<Image> \\
    & ... & ... \\
    \bottomrule
  \end{tabular}
  \caption{A comparison between normal stream processing operators and our
    degradation operators. \texttt{Vec<T>} represents a list of elements of type
    T. \texttt{Option<T>} indicates an optional element of type T which is
    either present \texttt{Some(T)} or absent \texttt{None}.}
  \label{tab:operators}
\end{table*}

\sysname{} offers a special operator \texttt{maybe} to annotate degradation
operations. Developers use \texttt{maybe} with concrete operations that affect
the data size as well as data fidelity: these are the knobs tunable by the
system to explore performance trade-offs. The basic form of \texttt{maybe}
operator takes two arguments: a knob and a function. The knob is a vector of
values with type $T$ that encodes possible levels of degradation. The function
performs the actual operation. By default, the function needs to satisfy the
type constrain: $f(T, I) \Rightarrow I$. That is, only operations that does not
alter the type of the data stream are allowed. In this way, downstream operators
can work without the need to know if the degradation is in effect or not. A
relaxed version, \texttt{maybe\_with\_default}, exists for flexibility; but care
should be taken to use.

We illustrate the idea of \texttt{maybe} operator with an example that quantizes
a stream of integers in the Rust programming language. One possible
implementation is as follows:

\vspace{-2pt}
\begin{lstlisting}
    let quantized_stream = vec![1, 2, 3, 4].into_stream()
        .maybe(vec![2, 4], |k, p| p.wrapping_div(k));
        .collect();
\end{lstlisting}

The code creates a stream of integers [1, 2, 3, 4] and pass it through a
degradation operation. In this example, the knob is chosen as [2, 4]. The
degradation function performs a wrapping (modular) division where the divisor is
the knob. Depending on the quantization level, the output will be either [1, 2,
3, 4] (no degradation), [0, 1, 1, 2] (k=2), or [0, 0, 0, 1] (k=4). The
\texttt{collect} method will run this stream and hold the output in a vector.
If the stream is subsequently encoded (e.g. run-length encoding as in
JPEG~\cite{wallace1992jpeg}) for transmission, the data size will change
according to the level of degradation.

Based on the \texttt{maybe} primitive, one can implement wrappers for common
degradation operations. For example, \texttt{maybe\_skip} will optionally
subsample a stream; and \texttt{maybe\_downsample} can adjust the image
resolution to a configured target. \sysname{} provides a few such operations as
libraries for developers.

Using our APIs, the example mentioned in \autoref{sec:manual-policy} can now be
implemented as follows:

\vspace{-2pt}
\begin{lstlisting}[caption={Video Processing Example}, label={lst:ex}]
   let app = Camera::new((1920, 1080, 30))
      .maybe_downsample(vec![(1600, 900), (1280, 720)])
      .maybe_skip(vec![2, 5])
      .map(|frame| frame.show())
      .compose();
\end{lstlisting}

This snippet first instantiate a \texttt{Camera} source, which produces
\texttt{Stream<Image>} with 1920x1080 resolution and 30 FPS. Two degradation
operations are chained after the source: one that downsample the resolution to
either 1600x900 or 1280x720; the other skip the frame with a parameter of 2 or
5, resulting in 30/(2+1)=10 FPS or 30/(5+1)= 6 FPS. After the degradation,
images are shown on the display. In practice, further processing (such as
encoding and transmission) operators can be chained.

While \maybe{} itself has simplified the specification of degradation, the exact
level of degradation has to be known for precise rate adjustment at runtime. The
second stage of \sysname{} performs automatic profiling.

%%% Local Variables:
%%% mode: latex
%%% TeX-master: "sosp17"
%%% End:
