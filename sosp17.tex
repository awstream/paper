\documentclass[sigplan, anonymous, review]{acmart}

% Removes citation information below abstract
\settopmatter{printacmref=false}
% removes footnote with conference information in first column
\renewcommand\footnotetextcopyrightpermission[1]{}
%% removes running headers
\pagestyle{plain}

\usepackage{booktabs} % For formal tables

% Copyright
\setcopyright{none}
%\setcopyright{acmcopyright}
%\setcopyright{acmlicensed}
%\setcopyright{rightsretained}
%\setcopyright{usgov}
%\setcopyright{usgovmixed}
%\setcopyright{cagov}
%\setcopyright{cagovmixed}

%% Below is commented out for submission
% % DOI
% \acmDOI{10.475/123_4}

% % ISBN
% \acmISBN{123-4567-24-567/08/06}

% % Conference
% \acmConference[SOSP'17]{}{Oct. 2017}{Shanghai, China}
% \acmYear{2017}
% \copyrightyear{2017}
% \acmPrice{15.00}
% \acmBadgeL[http://ctuning.org/ae/ppopp2016.html]{ae-logo}

\begin{document}
\title{Resilient Wide-area Stream Processing \\
  with Online Learning and Runtime Adaptation}

% The default list of authors is too long for headers}
\renewcommand{\shortauthors}{B. Zhang et al.}

\begin{abstract}
  Paper abstract.
\end{abstract}

%
% The code below should be generated by the tool at
% http://dl.acm.org/ccs.cfm
% Please copy and paste the code instead of the example below.
%
\begin{CCSXML}
<ccs2012>
  <concept>
    <concept_id>10010520.10010521.10010537</concept_id>
    <concept_desc>Computer systems organization~Distributed architectures</concept_desc>
    <concept_significance>300</concept_significance>
  </concept>
  <concept>
    <concept_id>10010520.10010570.10010574</concept_id>
    <concept_desc>Computer systems organization~Real-time system architecture</concept_desc>
    <concept_significance>300</concept_significance>
  </concept>
  <concept>
    <concept_id>10002951.10003227.10003251.10003255</concept_id>
    <concept_desc>Information systems~Multimedia streaming</concept_desc>
    <concept_significance>300</concept_significance>
  </concept>
  <concept>
    <concept_id>10002951.10003227.10010926</concept_id>
    <concept_desc>Information systems~Computing platforms</concept_desc>
    <concept_significance>300</concept_significance>
  </concept>
</ccs2012>
\end{CCSXML}

\ccsdesc[300]{Computer systems organization~Distributed architectures}
\ccsdesc[300]{Computer systems organization~Real-time system architecture}
\ccsdesc[300]{Information systems~Multimedia streaming}
\ccsdesc[300]{Information systems~Computing platforms}

\keywords{adaptation, learning, stream processing}

%% Below is the code to create the teaser
% \begin{teaserfigure}
%   \includegraphics[width=\textwidth]{sampleteaser}
%   \caption{This is a teaser}
%   \label{fig:teaser}
% \end{teaserfigure}

\maketitle

\section{Introduction}

The \textit{proceedings} are the records of a conference.  ACM seeks to give
these conference by-products a uniform, high-quality appearance.  To do this,
ACM has some rigid requirements for the format of the proceedings documents:
there is a specified format (balanced double columns), a specified set of fonts
(Arial or Helvetica and Times Roman) in certain specified sizes, a specified
live area, centered on the page, specified size of margins, specified column
width and gutter size.

Dolly~\cite{ananthanarayanan2013effective}.

\bibliographystyle{ACM-Reference-Format}
\bibliography{sosp17}

\end{document}

%%% Local Variables:
%%% mode: latex
%%% TeX-master: t
%%% End:
